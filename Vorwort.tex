% Persönlich
\addchap{Vorwort}
Die vorliegende Diplomarbeit wurde als Abschlussarbeit für die Höhere Technische Bundeslehr- und Versuchsanstalt Dornbirn erstellt. Nach unserer Ausbildung im Betriebsinformatik Zweig bildet die Diplomarbeit den Abschluss und gibt uns die Möglichkeit unser über die letzten Jahre angeeignetes Wissen in der Praxis anzuwenden.

Unser Projektteam fand sich in der Klasse zusammen und entschied sich, gemeinsam an dieser Diplomarbeit zu arbeiten, weil wir uns nicht nur gut in den zahlreichen Bereichen der Informatik ergänzen, sondern auch eine gute Beziehung zueinander haben.

Die Wahl des Themas für unsere Diplomarbeit war ein Prozess, der mehrere Kontakte zu verschiedenen Firmen beinhaltete. Dabei lernten wir die unterschiedlichen Unternehmen und Personen kennen und tauschten viele Projektideen aus.

Nach dem Vergleichen und Abwägen aller Optionen fiel die Entscheidung schlussendlich auf die Projektidee der Walter Bösch GmbH \& Co. KG. Die Idee zur Entwicklung einer Universalanzeige für \acl{rltanlagen} sprach jedes Teammitglied an, da es eine gute Mischung aus Hardware- und Softwarekomponenten beinhaltete, was uns ein großes Anliegen war. Außerdem verstärkte die renommierte Stellung von Bösch (ehem. als Heizbösch bekannt) als vorarlberger Unternehmen unsere Überzeugung. Die positive Erfahrung unseres Projektleiters, \mangeng, der zuvor bereits ein Praktikum bei Bösch absolviert hatte und dessen persönliche Eindrücke über das ansprechende Arbeitsklima trugen ebenfalls maßgeblich zur Entscheidung bei.

Diese Diplomarbeit reflektiert nicht nur unsere technischen Fähigkeiten, sondern auch die Fähigkeit, als Team zusammenzuarbeiten und Lösungen für Herausforderungen zu finden. Wir möchten uns bei allen, die uns auf diesem Weg unterstützt haben, herzlich bedanken.