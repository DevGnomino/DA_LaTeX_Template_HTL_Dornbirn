% Anleitung:
% https://en.wikibooks.org/wiki/LaTeX/Glossary

% BEFEHLE FÜR GLOSSAREINTRÄGE.

% \gls{<label>} 
% prints the term associated with <label> passed as its argument.

% \glspl{<label>}
% prints the plural of the defined term.

% \Gls{<label>}
% prints the singular form with the first character converted to upper case.

% \Glspl{<label>}
% prints the plural form with first character converted to upper case.

% BEFEHLE FÜR ACRONYME:

% \acrlong{<label>}
% long version of an acronym

% \acrfull{<label>}
% print the long version of an acronym and the abbreviation

% \acrshort{<label>}
% print the abbreviation


%Beispiele:
%\newacronym{vm}{VM}{Virtuelle Maschine}

\newglossaryentry{latex}
{
	name=latex,
	description={LaTeX (short for Lamport TeX) is a document preparation system. The user has to think about only the content to put in the document and the software will take care of the formatting}
}

\newglossaryentry{glsy}
{
	name=glossary,
	description={Acronyms and terms which are generally unknown or new to common readers}
}

\newglossaryentry{gls_gpio}
{
	name=GPIO,
	description={Als GPIO (Kurzform für General-Purpose Input-Output) bezeichnet man die Pins auf einem Mikrocontroller. Diese können elektrische Signale senden und empfangen, sind aber nicht für einen spezifischen Gebrauch entwickelt}
}

\newglossaryentry{hdmi}
{
	name=HDMI,
	description={HDMI (Kurzform für High-Definition Multimedia Interface) ist eine Multimedia-Schnittstelle für die Übertragung von Audio- und Videosignalen}
}

\newglossaryentry{osi}
{
	name=OSI,
	description={Das OSI (Open System Interconnection) Modell bildet die Netzwerkkommunikation in 7 Schichten ab. \cite[vgl.][]{Schnabel_osi:o.J.}}
}

\newglossaryentry{parity}
{
	name=Parität,
	description={Gibt an ob eine Zahl gerade oder ungerade ist, also ob sie ohne Rest durch 2 teilbar ist. \cite[vgl.][]{Parity_Mathematik:o.J.}}
}

\newglossaryentry{gls_tcp}
{
	name=TCP,
	description={Der TCP (Transmission Control Protocol) Standard definiert, wie eine Netzwerkkonversation aufgebaut und aufrechterhalten wird, über die Anwendungen Daten austauschen können. \cite[vgl.][]{Lutkevich:2021}}
}

\newglossaryentry{gls_rtu}
{
	name=RTU,
	description={Siehe Kapitel \ref{modbus_uebertragungsarten} \nameref{modbus_uebertragungsarten}}
}

\newglossaryentry{gls_ascii}
{
	name=ASCII,
	description={Der American Standard Code for Information Interchange ist ein Zeichensatz, der mit 7-Bits die meisten Zeichen einer US-Amerikanischen Computertastatur darstellen kann. \cite[vgl.][]{seo_kueche_ascii:o.J.}}
}

\newglossaryentry{client}
{
	name=Client,
	description={Siehe Kapitel \ref{modbus_funktionsweise} \nameref{modbus_funktionsweise}}
}

\newglossaryentry{server}
{
	name=Server,
	description={Siehe Kapitel \ref{modbus_funktionsweise} \nameref{modbus_funktionsweise}}
}

\newglossaryentry{gls_pdu}
{
	name=PDU,
	description={Siehe Kapitel \ref{modbus_funktionsweise} \nameref{modbus_funktionsweise}}
}

\newglossaryentry{gls_adu}
{
	name=ADU,
	description={Siehe Kapitel \ref{modbus_funktionsweise} \nameref{modbus_funktionsweise}}
}

\newglossaryentry{xor}
{
	name=XOR,
	description={XOR ist eine binäre Rechenoperation, bei der zwei Bits miteinander verglichen werden. Das resultierende Bit ist 1, wenn genau eines der verglichenen Bits 1 ist}
}

\newglossaryentry{gls_crc}
{
	name=CRC,
	description={Siehe Kapitel \ref{modbus_uebertragungsarten} \nameref{modbus_uebertragungsarten}}
}

\newglossaryentry{gls_lrc}
{
	name=LRC,
	description={Siehe Kapitel \ref{modbus_uebertragungsarten} \nameref{modbus_uebertragungsarten}}
}

\newglossaryentry{cr}
{
	name=CR,
	description={\gls{gls_ascii} Kontrollzeichen. Bewegt den Cursor an den Anfang der Zeile. \cite[vgl.][]{Mozilla_CRLF:2023}}
}

\newglossaryentry{lf}
{
	name=LF,
	description={\gls{gls_ascii} Kontrollzeichen. Bewegt den Cursor eine Zeile nach unten. \cite[vgl.][]{Mozilla_CRLF:2023}}
}

\newglossaryentry{gls_uart}
{
	name=UART,
	description={UART (Universal Asynchronous Receiver Transmitter) ist ein Protokoll für den seriellen Datenaustausch. UART benutzt zwei Drähte. \cite[vgl.][]{rohde:o.J.}}
}

\newglossaryentry{gls_minimalmodbus}
{
	name=Minimalmodbus,
	description={Siehe Kapitel \ref{minimalmodbus} \nameref{minimalmodbus}}
}

\newglossaryentry{gls_daemon}
{
	name=Daemon,
	description={Siehe Kapitel \ref{auslesen_rlt_parameter} \nameref{auslesen_rlt_parameter}}
}

\newglossaryentry{gls_thread}
{
	name=Thread,
	description={Mit Threads können asynchron Funktionen ausgeführt werden. Wenn ein neuer Thread gestartet wird, wird der Programmcode darin parallel zum Main Thread ausgeführt}
}

\newglossaryentry{gls_cli}
{
	name=CLI,
	description={Mit der CLI (Command Line Interface) können über textuelle Befehle mit dem Computer interagiert, also Programme ausgeführt und Dateien verwaltet werden. \cite[vgl.][]{loshin_gillis:2022}}
}

\newglossaryentry{vollduplex}
{
	name=Vollduplex,
	description={Siehe Kapitel \ref{rs485} \nameref{rs485}}
}

\newglossaryentry{halbduplex}
{
	name=Halbduplex,
	description={Siehe Kapitel \ref{rs485} \nameref{rs485}}
}

\newglossaryentry{gls_baudrate}
{
	name=Baudrate,
	description={Siehe Kapitel \ref{baud_rate} \nameref{baud_rate}}
}

% (Kapitel auslesen über Messwerte)
%minimalmodbus (Kapitel minimalmodbus / anders wo)
%Daemon (Kapitel auslesen über Messwerte)
%Thread (Kapitel auslesen über Messwerte)
%CLI (Kapitel Config Files übertragen

\newglossaryentry{zuluft}
{
	name=Zuluft,
	description={Bei der Zuluft handelt es sich um die Luft welche im Lüftungsgerät schon behandelt (gefiltert) wurde}
}

\newglossaryentry{abluft}
{
	name=Abluft,
	description={Bei der Abluft handelt es sich um die Luft welche aus dem Innenraum (Gebäude) abgesaugt wird}
}

\newglossaryentry{fortluft}
{
	name=Fortluft,
	description={Bei der Fortluft handelt es sich um die Luft welche aus dem Gebäude und der Lüftungsanlage nach draußen in die Umwelt geleitet wird}
}

\newglossaryentry{aussenluft}
{
	name=Außenluft,
	description={Bei der Außenluft handelt es sich um die Luft aus der Umwelt, welche noch unbehandelt (nicht gefiltert) in die Lüftungsanlage eingesaugt wird}
}

\newglossaryentry{tdot}
{
	name=TdoT,
	description={Der Tdot ist der Tag der offenen Tür der HTL Dornbirn. An diesem Tag wurde ein Lüftungsgerät der Firma Bösch ausgestellt inkl. RLT Anzeige}
}

\newglossaryentry{qbm}
{
	name=QBM,
	description={Eine Serie von Luftdruckfühlern der Marke Siemens. Im Falle der Diplomarbeit ist das Modell QBM9711 in Verwendung, welches über eine I/O-Erweiterung verfügt, an die externe Sensorik angeschlossen werden kann bzw. mit der externe Geräte gesteuert werden können}
}

\newglossaryentry{cprogrammiersprache}
{
	name=C,
	description={C ist eine prozedurale Programmierspache, die in den 1970er Jahren entwickelt wurde und wegen ihrer Schnelligkeit und Effizienz heutzutage noch Verwendung findet. Außerdem basieren viele Programmiersprachen, wie \zB Python und C++, auf C}
}

\newglossaryentry{toolkit}
{
	name=Toolkit,
	description={Als Toolkit wird eine Sammlung von Werkzeugen bezeichnet, die zu einem bestimmten Zweck dienen, wie z.B. der Softwareentwicklung}
}

\newglossaryentry{modbus}
{
	name=Modbus,
	description={Siehe Kapitel \ref{modbus_kapitel} \nameref{modbus_kapitel}}
}

\newglossaryentry{gls_rs485}
{
	name=RS485,
	description={Siehe Kapitel \ref{rs485} \nameref{rs485}}
}

\newglossaryentry{gls_json}
{
	name=JSON,
	description={Siehe Kapitel \ref{json_kapitel} \nameref{json_kapitel}}
}

\newglossaryentry{gls_csv}
{
	name=CSV,
	description={Siehe Kapitel \ref{csv_kapitel} \nameref{csv_kapitel}}
}

\newglossaryentry{gls_ssh}
{
	name=SSH,
	description={Secure Shell ist ein Netzwerkprotokoll, das verwendet wird, um sich sicher über ein Netzwerk mit einem anderen Gerät zu verbinden und auf diesem Gerät, aus der Ferne, Operationen auszuführen}
}

\newglossaryentry{gls_vnc}
{
	name=VNC,
	description={Mittels Virtual Network Computing (VNC) ist es möglich, sich, aus der Ferne, mit einem anderen Gerät zu verbinden. Dabei wird der Bildschirm dieses Geräts auf dem lokalen Gerät angezeigt und gleichzeitig die Tastatur- und Mauseingaben des lokalen Geräts an das entfernte Gerät gesendet}
}

\newglossaryentry{gls_python}
{
	name=Python,
	description={Siehe Kapitel \ref{python_kapitel} \nameref{python_kapitel}}
}

\newglossaryentry{gls_ctk}
{
	name=CustomTkinter,
	description={Siehe Kapitel \ref{ctk_kapitel} \nameref{ctk_kapitel}}
}

\newglossaryentry{gls_tk}
{
	name=Tkinter,
	description={Siehe Kapitel \ref{tk_kapitel} \nameref{tk_kapitel}}
}

\newglossaryentry{kapazitiv}
{
	name=Kapazitiv,
    	text=kapazitive,
	description={Ein kapazitives Display ermittelt die Position der Berührung durch Veränderung eines elektrischen Feldes. Dadurch kann genau ermittelt werden, wo die Berührung stattgefunden hat}
}

\newglossaryentry{resistiv}
{
	name=Resistiv,
    	text=resistive,
	description={Ein resistives Display ist eine Art berührungsempfindliches Display, welches den auf den Bildschirm ausgeübten Druck erkennt}
}

\newglossaryentry{image}
{
	name=Image,
	description={Als Image wird ein Speicherabbild eines gesamten Computers verstanden. Mit einem Image können Computer einfach auf einen bestimmten Stand zurückgesetzt werden oder neue Computer mit dem gleichen Stand aufgesetzt werden}
}

\newglossaryentry{gls_vm}
{
	name=VM,
	description={Eine virtuelle Maschine (VM) ist ein virtueller Computer, der wie ein eigenständiger Computer funktioniert und mithilfe eines Hypervisors auf einem physischen Computer läuft. Der Hypervisor ist eine Software, der die Ressourcen des physischen Computers für die virtuelle Maschine bereitstellt.}
}

