% Anleitung:
% https://en.wikibooks.org/wiki/LaTeX/Glossary

% BEFEHLE FÜR GLOSSAREINTRÄGE.

% \gls{<label>} 
% prints the term associated with <label> passed as its argument.

% \glspl{<label>}
% prints the plural of the defined term.

% \Gls{<label>}
% prints the singular form with the first character converted to upper case.

% \Glspl{<label>}
% prints the plural form with first character converted to upper case.

% BEFEHLE FÜR ACRONYME:

% \acrlong{<label>}
% long version of an acronym

% \acrfull{<label>}
% print the long version of an acronym and the abbreviation

% \acrshort{<label>}
% print the abbreviation


%Beispiele:
\newacronym{vm}{VM}{Virtuelle Maschine}

\newglossaryentry{latex}
{
	name=latex,
	description={LaTeX (short for Lamport TeX) is a document preparation system. The user has to think about only the content to put in the document and the software will take care of the formatting. }
}

\newglossaryentry{glsy}
{
	name=glossary,
	description={Acronyms and terms which are generally unknown or new to common readers.}
}

\newglossaryentry{gpio}
{
	name=GPIO,
	description={Als GPIO (Kurzform für General-Purpose Input-Output) bezeichnet man die Pins auf einem Mikrocontroller. Diese können elektrische Signale senden und empfangen, sind aber nicht für einen spezifischen Gebrauch entwickelt}
}

\newglossaryentry{hdmi}
{
	name=HDMI,
	description={HDMI (Kurzform für High-Definition Multimedia Interface) ist eine Multimedia-Schnittstelle für die Übertragung von Audio- und Videosignalen}
}

\newglossaryentry{osi}
{
	name=OSI,
	description={OSI (Open System Interconnection) ist die Internationale Organisation für Normung, als Grundlage für die Bildung von offenen Kommunikationsstandards}
}

\newglossaryentry{parity}
{
	name=Parität,
	description={Gibt an ob eine Zahl gerade oder ungerade ist}
}

\newglossaryentry{gls_tcp}
{
	name=TCP,
	description={TCP (Transmission Control Protocol) ist ein Standard, der definiert, wie eine Netzwerkkonversation aufgebaut und aufrechterhalten wird, über die Anwendungen Daten austauschen können}
}

\newglossaryentry{gls_rtu}
{
	name=RTU,
	description={RTU (Remote Terminal Unit)}
}

\newglossaryentry{gls_ascii}
{
	name=ASCII,
	description={ASCII ()}
}

\newglossaryentry{client}
{
	name=Client,
	description={}
}

\newglossaryentry{server}
{
	name=Server,
	description={}
}

\newglossaryentry{gls_pdu}
{
	name=PDU,
	description={}
}

\newglossaryentry{gls_adu}
{
	name=ADU,
	description={}
}

\newglossaryentry{xor}
{
	name=XOR,
	description={}
}

\newglossaryentry{gls_crc}
{
	name=CRC,
	description={\nameref{modbus_uebertragungsarten}}
}

\newglossaryentry{gls_lrc}
{
	name=LRC,
	description={\nameref{modbus_uebertragungsarten}}
}

\newglossaryentry{cr}
{
	name=CR,
	description={ASCII Kontrollzeichen. Bewegt den Cursor an den Anfang der Zeile}
}

\newglossaryentry{lf}
{
	name=LF,
	description={ASCII Kontrollzeichen. Bewegt den Cursor eine Zeile nach unten}
}

\newglossaryentry{zuluft}
{
	name=Zuluft,
	description={Bei der Zuluft handelt es sich um die Luft welche im Lüftungsgerät schon behandelt (gefiltert) wurde}
}

\newglossaryentry{abluft}
{
	name=Abluft,
	description={Bei der Abluft handelt es sich um die Luft welche aus dem Innenraum (Gebäude) abgesaugt wird}
}

\newglossaryentry{fortluft}
{
	name=Fortluft,
	description={Bei der Fortluft handelt es sich um die Luft welche aus dem Gebäude und der Lüftungsanlage nach draußen in die Umwelt geleitet wird}
}

\newglossaryentry{aussenluft}
{
	name=Außenluft,
	description={Bei der Außenluft handelt es sich um die Luft aus der Umwelt, welche noch unbehandelt (nicht gefiltert) in die Lüftungsanlage eingesaugt wird}
}

\newglossaryentry{tdot}
{
	name=TdoT,
	description={Der Tdot ist der Tag der offenen Tür der HTL Dornbirn. An diesem Tag wurde ein Lüftungsgerät der Firma Bösch ausgestellt inkl. RLT Anzeige}
}

\newglossaryentry{qbm}
{
	name=QBM,
	description={Eine Serie von Luftdruckfühlern der Marke Siemens. Im Falle der Diplomarbeit ist das Modell QBM9711 in Verwendung, welches über eine I/O-Erweiterung verfügt, an die externe Sensorik angeschlossen werden kann bzw. mit der externe Geräte gesteuert werden können}
}
