\subsection{Was ist \acs{csv}?}
\acf{csv} ist ein systemunabhängiges Format für Klartextdateien. \acs{csv} Dateien dienen zum Speichern und Übertragen von strukturierten Daten, hauptsächlich Tabellen oder Listen, wobei durch die Verkettung von mehreren \acs{csv} Dateien oder mithilfe von zusätzlichen Regeln auch verschachtelte Objekte gespeichert werden können. Die hauptsächlichen Anwendungsbereiche von \acs{csv} Dateien sind zum Importieren und Exportieren von Daten aus Datenbanken oder die Migration von Tabellendaten zwischen Programmen. \cite{FuchsMediaSolutions:o.J.}

Die erste Verwendung des Datenformates geht auf 1972 zurück, wo es vom IBM FORTRAN IV (H Extended) Compiler \cite{IBM:1972} unterstützt wurde. Trotz der langen Existenz gibt es gegenwärtig für \acs{csv} keine formelle Spezifikation. Mit dem \acs{rfc} 4180 \cite{Shafranovich:2005} aus dem Jahre 2005 existiert ein erster Versuch einer inoffiziellen Definition, welche mittlerweile weit verbreitet ist. Durch dieses Dokument wird das \acf{mime} "text/csv"\ für das \acs{csv} Format registriert. Es folgen die wesentlichen Merkmale aus der Definition des \acs{rfc} 4180:

\begin{enumerate}{}
	
	\item Jedem Datensatz steht eine Zeile zu, die mit einem Zeilenumbruch (\ac{crlf}) beendet wird. Ein Zeilenumbruch am Ende des letzten Datensatzes ist optional. \zB 
	\begin{lstlisting}
	aaa, bbb, ccc CRLF
	xxx, yyy, zzz CRLF
	\end{lstlisting}
	oder
	\begin{lstlisting}
	aaa, bbb, ccc CRLF
	xxx, yyy, zzz
	\end{lstlisting}
	
	\item Am Anfang eines \acs{csv} Dokumentes kann es eine Kopfzeile geben. Diese hat das Format eines normalen Datensatzes und beinhaltet Namen für die Spalten (Felder). Die Anzahl der Spalten sollte für Kopfzeile und Datensätze gleich sein. \zB
	\begin{lstlisting}
	spaltenname_1, spaltenname_2, spaltenname_3 CRLF
	aaa, bbb, ccc CRLF
	xxx, yyy, zzz CRLF
	\end{lstlisting}
	
	\item Sowohl in den Datensätzen als auch in der Kopfzeile kann es eine oder mehrere Spalten geben, die jeweils immer durch einen Beistrich (\acs{engl} comma) separiert werden. Am Ende einer Zeile bedarf es keines Beistrichs. Abstände sind Teil eines Feldes und müssen berücksichtigt werden. (Anmerkung: Auch wenn das eigentliche Format Beistriche für die Feldtrennung vorsieht, werden oft andere Zeichen wie \zB Strichpunkte (Semikolons) verwendet.)
	
	\item Jedes Feld kann in Anführungszeichen eingeschlossen sein und darf dann auch Beistriche, Zeilenumbrüche oder Anführungszeichen beinhaltet. \zB
	\begin{lstlisting}
	"aaa","b CRLF
	bb","ccc" CRLF
	xxx,yyy,zzz
	\end{lstlisting}
	
\end{enumerate}

Da es bei \acs{csv} keine festen Vorgaben beim Datenformat gibt, ist es die Verantwortung der Benutzer sich auf eine Formatierung zu einigen. Außerdem führt das zu Problemen bei Zeit- und Datumsangaben oder Verwendung von Sonderzeichen. Eine weitere Hürde ist die fehlende explizite Angabe des verwendeten Zeichensatzes, womit \zB Umlaute fehlerhaft dargestellt werden. \cite{FuchsMediaSolutions:o.J.}

\subsection{Was ist \acs{json}?}
\acf{json} ist ein Textformat zum Speichern von strukturierten Daten. Es ist so konzipiert, dass es sowohl für Menschen einfach zu lesen und schreiben als auch für Maschinen einfach zu parsen und generieren ist. \acs{json} stammt von JavaScript, ist aber programmiersprachenunabhängig und folgt vielen Konventionen der C-basierten Sprachen, was es gut für den Datenaustausch zwischen Programmiersprachen macht. Die zwei wichtigsten Strukturen auf denen \acs{json} aufbaut, sind eine geordnete Liste von Werten und Name/Wert (Key/Value) Paare. \cite{json_org:o.J.} 

\acs{json} wird derzeit von zwei Spezifikation definiert, ECMA-404 \cite{ECMA:2017} und RFC 8259 \cite{Bray:2017}. Dabei ist das Ziel, dass sich nur die Beschreibung des Formats unterscheidet, aber die \acs{json} Grammatik (Syntax) beider Spezifikationen ident ist. Folgend wird sich auf die Beschreibung der offiziellen JSON.org Website \cite{json_org:o.J.} und somit auf ECMA-404 \cite{ECMA:2017} bezogen.

Für die \acs{json} Syntax werden folgende Zeichen verwendet:
\begin{itemize}
 \item Beistriche \lstinline|,|
 \item Doppelpunkte \lstinline|:|
 \item Eckige Klammern \lstinline|[ ]|
  \item Geschwungene Klammern \lstinline|{ }|
\end{itemize}


In \acs{json} gibt es:
\begin{itemize}
	\item \textbf{Werte:} Diese sind entweder vom Typ "'object"', "'string"' (Zeichenketten), "'array"', "'number"' oder haben den Wert "'false"', "'true"', oder "'null"'.
		
	\item \textbf{Objekte:} description
	
	\item \textbf{Arrays:} description
	
	\item \textbf{Zeichenketten} description
\end{itemize}

\subsection{\acs{json} und \acs{csv} - Vergleich und Selektion}
\cite{SQLizer:2017}
