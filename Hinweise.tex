\addchap{Begriffserklärung} \label{begriffserklaerung}

\noindent In dieser Diplomarbeit werden zur einfacheren Lesbarkeit manche Bezeichnungen verkürzt geschrieben \bzw nicht voll ausgeschrieben. In Tab.~\ref{tab:begriffserklaerung} sind oft verwendete Abkürzungen mit den jeweils dazugehörenden langen Bezeichnungen zu sehen. Außerdem sind jeglich verwendete Akronyme am Ende der Diplomarbeit im Abkürzungsverzeichnis zu finden. 
\begin{table}[h]
	\caption{Begriffserklärung \label{tab:begriffserklaerung}}
	\begin{tabularx}{\textwidth}{@{}c|c|X@{}}
		\toprule
		\textbf{Kurze Bezeichnung} & \textbf{Lange Bezeichnung} & \textbf{Beschreibung} \\
		\midrule
        \acs{rltanlage} & \Acl{rltanlage} &  Produkt, das die Walter Bösch GmbH \& Co KG herstellt \\
		\acs{rltanzeige} & \acl{rltanzeige} &  Produkt, das im Rahmen dieser Diplomarbeit entwickelt werden soll \\
		Bösch & Walter Bösch GmbH \& Co KG & Projektauftraggeber dieser Diplomarbeit \\
		\bottomrule
	\end{tabularx}
\end{table}


%Modbus - Client/Server
In vielen älteren Dokumentation über das Modbus Protokoll finden sich veraltete Bezeichnungen. In dieser Diplomarbeit wird auf die alten Bezeichnungen verzichtet und die Neuen verwendet. In einzelnen Abbildungen können die veralteten Bezeichnungen jedoch noch zu finden sein (siehe Tab.~\ref{tab:modbus_bezeichnung}). 
\begin{table}[h]
	\caption{Modbus Bezeichnungen \label{tab:modbus_bezeichnung}}
	\begin{tabularx}{\textwidth}{@{}c|c|X@{}}
		\toprule
		\textbf{Neu} & \textbf{Veraltet} & \textbf{Beschreibung} \\
		\midrule
		Client & Master & Initialisiert die Kommunikation und kann Daten von den Servern anfordern \\
		Server & Slave & Wird vom Client angesprochen, um Daten zu senden und Handlungen auszuführen \\
		\bottomrule
	\end{tabularx}
\end{table}

\addchap{Hinweis zur Textformatierung}

\noindent Im Rahmen dieser Diplomarbeit ist es von essenzieller Bedeutung, eine kohärente Textformatierung zu wahren, um Lesbarkeit und Verständlichkeit zu fördern. Dabei wurde folgendes definiert:
\begin{itemize}
    \item Alle Fachbegriffe innerhalb des Textes werden konsequent \textit{kursiv} formatiert, um ihre Hervorhebung und klare Identifikation zu gewährleisten. Zusätzlich finden sich umfassende Erläuterungen zu sämtlichen Fachtermini im Glossar.

    \item Im linken Teil der Fußzeile ist in den relevanten Teilen der Diplomarbeit für jeden Abschnitt des Dokuments der Name des jeweiligen Verfassers zu finden.  
    
    \item In den Codeblöcken wird teils weitere Funktionalität zur besseren Übersichtlichkeit ausgelassen oder später beschrieben. Dies wird folgendermaßen gekennzeichnet: \begin{pythoncode}
#[Anmerkung zum ausgelassenen bzw. vereinfachten Code]
    \end{pythoncode}

    \item Wenn im Text Code beschrieben wird und Klassennamen, Funktionsnamen oder Variablennamen erwähnt werden, sind diese stets mit einer speziellen Formatierung hervorgehoben, wie \zB \enquote{Die \lstinline{Variable} wird für...}
\end{itemize}

%VIELLEICHT WEITERE SACHEN?

%ZITIERRICHTLINIEN?
%Die DIN~ISO~690~\cite{DIN-ISO-690:2013} gibt Hinweise zur vollständigen Quellenangabe.


