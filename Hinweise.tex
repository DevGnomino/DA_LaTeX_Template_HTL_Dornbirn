\addchap{Begriffserklärung}

% RLT Anlagen oder Lüftungsgeräte - Überhaupt das gleiche?

% Ventilatoren

In dieser Diplomarbeit werden zur einfacheren Lesbarkeit manche Bezeichnungen verkürzt geschrieben \bzw nicht voll ausgeschrieben. In Tab.~\ref{tab:begriffserklaerung} sind jeweils die langen Bezeichnungen mit den dazugehörigen Abkürzungen zu sehen:
\begin{table}[h]
	\caption{Begriffserklärung \label{tab:begriffserklaerung}}
	\begin{tabularx}{\textwidth}{@{}c|c|X@{}}
		\toprule
		\textbf{Kurze Bezeichnung} & \textbf{Lange Bezeichnung} & \textbf{Beschreibung (falls nötig)} \\
		\midrule
		\acs{rltanlage} & Universalanzeige für RLT-Anlagen &  Produkt, das mit dieser Diplomarbeit entwickelt werden soll \\
		Bösch & Walter Bösch GmbH \& Co KG & Projektauftraggeber dieser Diplomarbeit \\
		\bottomrule
	\end{tabularx}
\end{table}


%Modbus - Client/Server
In vielen älteren Dokumentation über das Modbus Protokoll finden sich veraltete Bezeichnungen. In dieser Diplomarbeit werden auf die alten Bezeichnungen verzichtet und die Neuen verwendet. In einzelnen Abbildungen können die veralteten Bezeichnungen jedoch noch zu finden sein (siehe Tab.~\ref{tab:modbus_bezeichnung}). 
\begin{table}[h]
	\caption{Modbus Bezeichnungen \label{tab:modbus_bezeichnung}}
	\begin{tabularx}{\textwidth}{@{}c|c|X@{}}
		\toprule
		\textbf{Neu} & \textbf{Veraltet} & \textbf{Beschreibung} \\
		\midrule
		Client & Master & Initialisiert die Kommunikation und kann Daten von den Servern anfordern \\
		Server & Slave & Wird vom Client angesprochen, um Daten zu senden und Handlungen auszuführen \\
		\bottomrule
	\end{tabularx}
\end{table}

\addchap{Hinweis zur Textformatierung}
%Hinweis zur Textformatierung (z.B. Fachbegriffe Kursiv schreiben)

%Noch entscheiden: CSV-Dateien, JSON-Dateien usw. mit oder ohne Bindestrich?

