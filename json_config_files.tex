\subsection{Dateikonzept} \label{json_config_files}
Für das Dateikonzept wurden drei Dateitypen auf Basis von \acs{json} Dateien entwickelt:

%Fenkart fragen, ob ich bei den Geräten auf seinen Teil verweisen kann
% Wer beschreibt baud_rate, register, adresse, function code \ref{modbus_funktionsweise}

\begin{enumerate}

	\item \textbf{Sensor-Konfigurationsdatei} (\enquote{sensors.json}): Eine Liste aller Sensoren (\zB LG-NI1000) mit der dazugehörigen Maßeinheit. Dieses muss nur verändert werden, wenn ein neuer Sensor eingebaut wird und dieser noch nicht im \enquote{sensors.json} steht.
	
	\item \textbf{Geräte-Konfigurationsdateien} (\zB \enquote{QBM97XX.json} oder \enquote{EBM.json}): Hier sind gerätespezifische Daten hinterlegt bzw. welche Ports vom Gerät verwendet werden, welche Einstellungen diese haben und im Falle des QBMs, welche möglichen Maßeinheiten die anzuschließenden Sensoren zurückgeben. Diese Dateien müssen nur verändert werden, falls an einem QBM z.B. ein neuer Sensor verbaut wird, der eine neue Maßeinheit zurückgibt. Der wahrscheinlichere Fall, ist aber, dass nicht eine vorhandene Datei verändert wird, sondern eine weitere Geräte-Konfigurationsdatei für ein neues Gerät (z.B. Ventilator einer neuen Firma) erstellt werden muss.
	
	\item \textbf{Haupt-Konfigurationsdatei} (\enquote{main\_config\_file.json}): Wie in Kapitel \ref{gui_design} beschrieben, werden zur übersichtlichen  Anzeige der Messwerte mehrere Abschnitte \bzw Seiten benötigt. Auf diesen Seiten werden jeweils sinngemäß Messwerte zusammengefasst. 
	
	Die Aufteilung der Seiten wird in der Haupt-Konfigurationsdatei gemacht. Dabei werden alle Seiten, die später auf der \acs{rltanzeige} angezeigt werden sollen, im \enquote{pages} Array definiert. Dazu werden unterschiedliche Parameter angegeben (siehe Tab. \ref{tab:pages_array_parameter}).

		\begin{table}[H]
			\caption{Parameter im \enquote{pages} Array}
			\label{tab:pages_array_parameter}
			\begin{tabularx}{\textwidth}{@{}lX|X@{}}
				\toprule
				\textbf{Name} & \textbf{Beschreibung} & \textbf{Beispiel} \\
				\midrule
				title      	& Beliebig auswählbarer Titel der Seite. &  Bsp
				\\
				sources 	& Array aller Messwerte, die auf einer Seite angezeigt werden sollen. Die Parameter, die in einem Objekt dieses Arrays benutzt werden, sind in Tab. \ref{tab:sources_array_parameter} zu sehen. & Bsp
				\\
				\bottomrule
			\end{tabularx}
		\end{table}
		
		\begin{table}[H]
			\caption{Parameter im \enquote{sources} Array}
			\label{tab:sources_array_parameter}
			\begin{tabularx}{\textwidth}{@{}lX|X@{}}
				\toprule
				\textbf{Name} & \textbf{Beschreibung} & \textbf{Beispiel} \\
				\midrule
				port                & Ein Array mit den Quellen, aus denen der Messwert ausgelesen wird. Dabei wird meistens nur ein Objekt mit \enquote{id} des Geräts und \enquote{port} (siehe Tab. \ref{tab:devices_array_parameter})angegeben. In manchen Fällen werden mehrere Quellen für abgeleitete Messwerte benötigt (siehe Kapitel \ref{python_functions}). In dem Fall werden dem Array weitere Objekte hinzugefügt. & Bsp \\
				description         & Beliebig auswählbare Bezeichnung für einen Messwert, die auf der \acs{rltanzeige} später angezeigt wird. & Bsp \\
				python\_function    & Optionaler Parameter, der angegeben wird, wenn der jeweilige Messwert abgeleitet berechnet werden muss. Weitere Erklärung in Kapitel \ref{python_functions}. & Bsp \\
				additional\_info    & Optionaler Parameter, der angegeben wird, falls bei der Python Funktion zusätzliche Informationen braucht. Weitere Erklärung in Kapitel \ref{python_functions}. & Bsp \\
				\bottomrule
			\end{tabularx}
		\end{table} 
		
	Ein Beispiel für einen Seiteneintrag im \enquote{pages} Array ist folgend zu sehen:
	
%	\jsonfile[firstline=2, lastline=12]{Code/main.py}
	\begin{jsoncode}
	"pages": [
		{
			"title": "Temperaturen QBM1",
			"sources": [
			{
				"port": [{ "QBM1": "AI1" }],
				"description": "Temperatur 1"
			},
			{
				"port": [{ "QBM1": "AI2" }],
				"description": "Temperatur 2"
			},
			...
			{
				"port": [{ "QBM1": "AO1" }],
				"description": "Klappe",
				"python_function": "flap_position"
			}
			]
		},
		...
	]
	\end{jsoncode}
	
		
	Parallel zum \enquote{pages} Array \bzw auf der gleichen Ebene der Haupt-Konfigurationsdatei gibt es ein \enquote{devices} Array. Darin werden die Komponenten der \acs{rltanlage} angegeben, von denen Messwerte ausgelesen werden sollen.
	
	\begin{table}[H]
		\caption{Parameter im \enquote{devices} Array}
		\label{tab:devices_array_parameter}
		\begin{tabularx}{\textwidth}{@{}lX|X@{}}
			\toprule
			\textbf{Name} & \textbf{Beschreibung} & \textbf{Beispiel} \\
			\midrule
			device     	&  & Bsp \\
			id         	&  & Bsp \\
			baud\_rate	&  & Bsp \\
			mbaddress	&  & Bsp \\
			parity	&  & Bsp \\
			stop\_bits	&  & Bsp \\
			zero\_based	&  & Bsp \\
			sensors	&  & Bsp \\
			\bottomrule
		\end{tabularx}
	\end{table} 
	
\end{enumerate}

%Bedienungsanleitung für Servicetechniker erwähnen
Dabei ist zu beachten:
Der Techniker bekommt einen Ordner, darin findet sich direkt eine Vorlage für die \enquote{main\_config\_file.json}-Datei, welche im Normalfall die einzige Datei sein sollte, die vom Techniker bearbeitet bzw. verändert wird.
In einem Unterordner (\enquote{devices}) sind dann die Geräte-Konfigurationsdateien und die Sensor-Konfigurationsdatei. Diese Dateien müssen, wie oben erwähnt, nur verändert werden, wenn noch nie verwendete Geräte oder Sensorik, wie z.B. eine neue Art von Ventilator oder Sensor, in der RLT-Anlage verbaut werden.
\gls{latex}




