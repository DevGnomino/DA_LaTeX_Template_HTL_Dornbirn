\label{python_functions}
Messwerte werden aus \gls{modbus} Registern der unterschiedlichen Komponenten einer \acs{rltanlage} bezogen.  Grundsätzlich wird jeder ausgelesene Wert direkt auf der \acs{rltanzeige} angezeigt, wobei Skalierung und Maßeinheit hinzugefügt werden. In einigen Fällen muss jedoch entweder ein Wert aus mehreren Messwerten abgeleitet werden, eine spezielle Umrechnung oder Ähnliches erfolgen, bevor der Wert auf der \acs{rltanzeige} dargestellt werden kann. In einem solchen Fall wird in der Haupt-Konfigurationsdatei im \enquote{pages} Array (vgl. Kapitel \ref{json_config_files}) der Name einer \gls{gls_python} Funktion angegeben, die ausgeführt werden muss, um den erwünschen Wert zu erhalten. 

Diese \gls{gls_python} Funktionen sind in der Datei \enquote{modbus\_functions.py} definiert, was den folgenden Vorteil bietet: Ein einfaches Hinzufügen neuer Komponenten, deren Messwerte eine spezielle Berechnung benötigen, ohne dass der restliche Code verändert werden muss. Bei der Integration einer neuen Komponente werden lediglich die nötigen Funktionen in der \enquote{modbus\_functions.py} Datei hinzugefügt und diese folgend in der Haupt-Konfigurationsdatei referenziert.

Eine solche \gls{gls_python} Funktion hat den folgenden grundlegenden Aufbau:
\begin{pythoncode}
def Funktionsname(sensors, additional_info):
	...	
	return Ergebnis
\end{pythoncode}

Jede Funktion erwartet die Übergabeparameter \enquote{sensors} und \enquote{additional\_info}. \enquote{sensors} ist eine Liste, die Instanzen der Klasse \lstinline{Sensor} beinhaltet. Eine Instanz der Klasse \lstinline{Sensor} wiederum enthält alle erforderlichen Informationen, um einen Wert aus einem \gls{modbus} Register auszulesen, wie es in Kapitel \ref{auslesen_rlt_parameter} beschrieben wird. \newline 
Der Parameter \enquote{additional\_info} ermöglicht die Übermittlung weiterer Informationen, falls diese bei der Berechnung oder Ableitung der Werte benötigt werden.

Im folgenden Abschnitt werden die bisher definierten \gls{gls_python} Funktionen beschrieben, eine weitere Erklärung zum Ablauf bei der Ausführung der \gls{gls_python} Funktionen ist in Kapitel \ref{auslesen_rlt_parameter} zu finden.


\paragraph{Funktion \enquote{standard}}
Diese Funktion wird standardmäßig ausgeführt, wenn in der Haupt-Konfigurationsdatei (vgl. Kapitel \ref{json_config_files}) der Parameter \enquote{python\_function} nicht angegeben wird \bzw keine besondere \gls{gls_python} Funktion angegeben wird. Die \lstinline{standard()} Funktion wird daher als einzige dieser Funktionen nie direkt in der Haupt-Konfigurationsdatei referenziert.

Wie im unten stehenden Code zu sehen ist, wird bei dieser Funktion nichts berechnet. Es wird lediglich die \lstinline{get_data_from_modbus()} Funktion der \lstinline{Sensor} Instanz aufgerufen, welche den Messwert aus dem jeweiligen \gls{modbus} Register ausliest und zurück gibt (weitere Erklärung in Kapitel \ref{auslesen_rlt_parameter}).

\begin{pythoncode}
def standard(sensors, additional_info):
	return sensors[0].get_data_from_modbus()
\end{pythoncode}


\paragraph{Funktion \enquote{calc\_rpm}}
Diese Funktion wird verwendet, um bei ebm-papst Ventilatoren die Soll- \bzw Ist-Drehzahl zu ermitteln. Dabei werden, wie im folgenden Code zu sehen ist, zwei Register ausgelesen. Daraufhin wird das Verhältnis der jeweiligen Drehzahl zur maximalen Drehzahl des Ventilators berechnet, das Ergebnis gerundet und in Prozent zurückgegeben.

\begin{pythoncode}
def calc_rpm(sensors, additional_info):
	max_value = sensors[1].get_data_from_modbus()
	rpm_value = (sensors[0].get_data_from_modbus() / 64000) * max_value
	ratio = rpm_value / max_value * 100.0
	return str(round(ratio, 1)) + " %"
\end{pythoncode}

Die Gleichung \eqref{glg:drehzahl_berechnung} für die Berechnung des Wertes \enquote{rpm\_value} stammt aus dem Datenblatt für ebm-papst Ventilatoren \cite[vgl.][118,122]{ebmpapst:2020}: 
\begin{equation}
	\text{rpm\_value}\left[\frac{1}{min}\right] = \frac{\text{Datenbytes}}{64000} \cdot \text{max\_value} \left[\frac{1}{min}\right]
	\label{glg:drehzahl_berechnung}
\end{equation} 

Eine Angabe in der Haupt-Konfigurationsdatei sieht wie folgt aus. Dabei ist zu sehen, dass im \enquote{port} Array zuerst die Quelle der Ist- \bzw Soll-Drehzahl (\enquote{RPMreal} \bzw \enquote{RPMtarget}) und als zweites die Quelle der maximalen Drehzahl (\enquote{RPMmax}). Es wird keine \enquote{additional\_info} angegeben.

\begin{jsoncode}
"sources": [
	{
		"port": [
			{"EBM1": "RPMreal"},
			{"EBM1": "RPMmax"}
		],
		"description": "Drehzahl Istwert",
		"python_function": "calc_rpm"
	},
	...
]
\end{jsoncode}