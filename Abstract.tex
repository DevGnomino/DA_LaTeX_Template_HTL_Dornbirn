% Vorwort übersetzt
\addchap{Abstract (DE)}
Das Ziel dieser Diplomarbeit ist die vereinfachte und übersichtliche Darstellung von Messwerten einer \acf{rltanlage} sowohl für Servicetechnikerinnen und Servicetechniker als auch Kundinnen und Kunden der Walter Bösch GmbH \& Co. KG. Dies wird durch die Entwicklung einer universellen Anzeige erreicht, die Daten über das \gls{modbus} Protokoll ausliest. 

Im ersten Teil dieser Arbeit werden zunächst grundlegende Begriffe erläutert, die für das Verständnis einer raumlufttechnischen Anlage und ihrer Messwerte wesentlich sind. Dies beinhaltet eine eingehende Betrachtung von Lüftungsgeräten sowie des \gls{modbus} Protokolls. Darüber hinaus erfolgt im ersten Teil die Beschreibung der Planung und Vorbereitung des Projekts. Dies umfasst die Hard- und Softwareevaluierung, dessen Ergebnis die Entscheidung für die Nutzung eines Raspberry PI in Kombination mit \gls{gls_python} und \gls{gls_json} ist. 
\newline Im zweiten Teil der Diplomarbeit steht die Umsetzung im Fokus. Hier wird auf das Design der \acf{gui}, die Konzeption der \gls{gls_json} Konfigurationsdateien, die Struktur des Programmcode und auf die Konfiguration des Raspberry PI eingegangen. 
\newline Zuletzt werden im letzten Teil der Arbeit der Prozess des Projektmanagements beschrieben und jegliche Projektmanagementpläne präsentiert.


\selectlanguage{english} 
\addchap{Abstract (EN)}














\selectlanguage{ngerman} 
