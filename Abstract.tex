% Vorwort übersetzt
\addchap{Abstract (DE)}
\noindent Das Ziel dieser Diplomarbeit ist die vereinfachte und übersichtliche Darstellung von Messwerten einer raumlufttechnischen Anlage (\acs{rltanlage}) sowohl für Servicetechnikerinnen und Servicetechniker als auch Kundinnen und Kunden der Walter Bösch GmbH \& Co. KG. Dies wird durch die Entwicklung einer universellen Anzeige erreicht, die Daten über das \gls{modbus} Protokoll ausliest. 

Im ersten Teil dieser Arbeit werden zunächst grundlegende Begriffe erläutert, die für das Verständnis einer raumlufttechnischen Anlage und ihrer Messwerte wesentlich sind. Dies beinhaltet eine eingehende Betrachtung von Lüftungsgeräten sowie des \gls{modbus} Protokolls. Darüber hinaus erfolgt im ersten Teil die Beschreibung der Planung und Vorbereitung des Projekts. Dies umfasst die Hard- und Softwareevaluation. Das Ergebnis hierbei ist die Entscheidung zur Nutzung eines Raspberry PIs in Kombination mit \gls{gls_python} und \gls{gls_json}. 

Im zweiten Teil der Diplomarbeit steht die Umsetzung im Fokus. Hier wird auf das Design der grafischen Benutzeroberfläche (\acs{gui}), die Konzeption der \gls{gls_json} Konfigurationsdateien, die Struktur des Programmcodes und auf die Konfiguration des Raspberry PIs eingegangen. 

Abschließend wird im letzten Teil der Arbeit der Prozess des Projektmanagements beschrieben und jegliche Projektmanagementpläne präsentiert.

Die Ergebnisse der Arbeit umfassen den funktionsfähigen Prototypen der \ac{rltanzeige}, die Software, eine umfassende Bedienungsanleitung für Endbenutzerinnen und Endbenutzer und eine Dokumentation für Servicetechnikerinnen und Servicetechniker.


\selectlanguage{english} 
\addchap{Abstract (EN)}
\noindent The objective of this diploma thesis is the simplified and clear visualization of measured values of a \acf{rltanlage-engl} for both service technicians and customers of Walter Bösch GmbH \& Co. KG. This is achieved by developing a versatile display that reads out data via the \gls{modbus} protocol. 

The first part of this thesis begins by explaining basic terms that are essential to understanding a \ac{rltanlage-engl} and its measured values. This includes an in-depth look at ventilation units and the \gls{modbus} protocol. In addition, the first part describes the planning and preparation for the project. This includes the hardware and software evaluation, the result of which is the decision to use a Raspberry PI in combination with \gls{gls_python} and \gls{gls_json}. 

The second part of the thesis focuses on the implementation. This is where the design of the \acf{gui-engl}, the conception of the \gls{gls_json} configuration files, the structure of the source code and the configuration of the Raspberry PI are discussed.

Finally, the last part of the thesis describes the project management process and presents various project management plans.

The results of this diploma thesis include a functioning prototype of the \ac{rltanzeige-engl}, the software, a comprehensive user manual for end users and documentation for service technicians.


\selectlanguage{ngerman} 
