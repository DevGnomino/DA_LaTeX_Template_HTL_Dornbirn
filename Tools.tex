\ifoot{\leftmark}
\chapter{Tools}
%Tools (z.B. Latex, Figma, Shortbus)
\noindent Zur Ausarbeitung dieser Diplomarbeit wurden viele unterschiedliche Programme und Tools verwendet. Es folgt eine kurze Auflistung dieser:
\begin{itemize}
	\item \textbf{\LaTeX}: Zur Erstellung dieses Dokuments \textit{(EINE KURZE WEITERE ERKLÄRUNG FOLGT!!!!!)}
	\item \textbf{Word} und andere Editoren: Zur Erstellung von Notizen
	\item \textbf{Excel}: Um die Bewertungsmatrix der Hardwareauswahl zu erstellen
	\item \textbf{Onedrive}: Um die Notizen und Dokumente zu speichern und untereinander zugänglich zu machen.
	\item \textbf{Github}: Damit wurde der Programmcode verwaltet und gespeichert.
	\item \textbf{Discord und Whatsapp}: Zur internen Kommunikation zwischen den Teammitgliedern.
	\item \textbf{Outlook und Teams}: Zur Kommunikation mit der Firma und dem Betreuungslehrer.
	\item \textbf{Shortbus}: Es handelt sich hierbei um ein Programm, mit dem über eine grafische Oberfläche Modbus Register ausgelesen und gesetzt werden können (https://sourceforge.net/projects/shortbusmodbusscanner/). Damit kann die Modbus Kommunikation getestet und die relevanten Register bestimmt werden.
	\item \textbf{Visual Studio Code und PyCharm}: Sind beides IDEs, mit denen das Python Programm der Anzeige entwickelt wurde.
	\item \textbf{Figma}: Siehe Kapitel \ref{figma_design} \nameref{figma_design}.
	\item \textbf{Zotero}: Es ist ein Literaturverwaltungsprogramm. Damit werden Internetquellen gesammelt, damit sie in Latex importiert werden können.
    \item \textbf{Raspberry PI Imager}:
    \item \textbf{balenaEtcher}:
\end{itemize}
