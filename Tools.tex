\ifoot{\leftmark}
\chapter{Tools}
%Tools (z.B. Latex, Figma, Shortbus)
\noindent Zur Ausarbeitung dieser Diplomarbeit wurden viele unterschiedliche Programme und Tools verwendet. Es folgt eine kurze Auflistung dieser:
\begin{itemize}
	\item \textbf{\LaTeX}: Zur Erstellung dieses Dokuments. \LaTeX \ ist eine freie Software mit der Texte formatiert werden können und ist eine Alternative zu Textverarbeitungsprogrammen wie Microsoft Word. Insbesondere ist \LaTeX \ ein weit verbreitetes System zur Erstellung von wissenschaftlichen Dokumenten. \cite[vgl.][]{latex_o.J.}
	\item \textbf{Word} und andere Editoren: Zur Erstellung firmeninterner Dokumente und Notizen für die spätere Dokumentation.
	\item \textbf{Excel}: Um die Bewertungsmatrix der Hardwareauswahl zu erstellen.
	\item \textbf{Onedrive}: Um die Notizen und Dokumente zu speichern und im Projektteam untereinander zugänglich zu machen.
	\item \textbf{Github}: Damit wird der Programmcode verwaltet und gespeichert.
	\item \textbf{Discord und Whatsapp}: Zur internen Kommunikation zwischen den Teammitgliedern.
	\item \textbf{Outlook und Teams}: Zur Kommunikation mit der Firma und dem Betreuungslehrer.
	\item \textbf{Shortbus}: Es handelt sich hierbei um ein Programm, mit dem über eine grafische Oberfläche \gls{modbus} Register ausgelesen und gesetzt werden können (https://sourceforge.net/projects/shortbusmodbusscanner/). Damit kann die \gls{modbus} Kommunikation getestet und die relevanten Register bestimmt werden.
	\item \textbf{Visual Studio Code und PyCharm}: Sind beides integrierte Entwicklungsumgebungen (IDEs), mit denen das \gls{gls_python} Programm der \ac{rltanzeige} entwickelt wurde.
	\item \textbf{Figma}: Zur Erstellung des Designs \bzw der \ac{gui} für die \ac{rltanzeige}. Siehe Kapitel \ref{figma_design} \nameref{figma_design}.
	\item \textbf{Zotero}: Es ist ein Literaturverwaltungsprogramm. Damit werden Internetquellen gesammelt, die später für den Import in \LaTeX \ exportiert werden können.
    \item \textbf{Raspberry PI Imager}: Um das Basis-\gls{image} auf die Raspberry PIs zur Entwicklung zu spielen.
    \item \textbf{balenaEtcher}: Um das fertige \gls{image} der \ac{rltanzeige} auf SD-Karten zu flashen.
\end{itemize}
