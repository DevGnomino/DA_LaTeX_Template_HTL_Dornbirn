\chapter{Tools}
%Tools (z.B. Latex, Figma, Shortbus)
Zur Ausarbeitung dieser Diplomarbeit wurden viele unterschiedliche Programme und Tools verwendet. Es folgt eine kurze Auflistung dieser:
\begin{itemize}
	\item Latex: Zur Erstellung dieses Dokuments
	\item Word und andere Editoren: Zur Erstellung von Notizen
	\item Excel: Um die Bewertungsmatrix der Hardwareauswahl zu erstellen
	\item Onedrive: Um die Notizen und Dokumente zu speichern und untereinander zugänglich zu machen.
	\item Github: Damit wurde der Programmcode verwaltet und gespeichert.
	\item Discord und Whatsapp: Zur internen Kommunikation zwischen den Teammitgliedern.
	\item Outlook und Teams: Zur Kommunikation mit der Firma und dem Betreuungslehrer.
	\item Shortbus: Es handelt sich hierbei um ein Programm, mit dem über eine grafische Oberfläche Modbus Register ausgelesen und gesetzt werden können (https://sourceforge.net/projects/shortbusmodbusscanner/). Damit konnte die Modbus Kommunikation getestet und die relevanten Register bestimmt werden.
	\item Visual Studio Code und PyCharm: Sind beides IDEs, mit denen das Python Programm der Anzeige entwickelt wurde.
	\item Figma: Ist eine Website mit dieser GUIs designt werden können. Damit wurde das Aussehen der \acs{gui} der Anzeige gestaltet.
	\item Zotero: Es ist ein Literaturverwaltungsprogramm. Damit wurden Internetquellen gesammelt, damit sie in Latex importiert werden können.
\end{itemize}
