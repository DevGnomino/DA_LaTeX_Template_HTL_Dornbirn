\ifoot{\leftmark}
\chapter{Tools}
%Tools (z.B. Latex, Figma, Shortbus)
\noindent Zur Ausarbeitung dieser Diplomarbeit wurden viele unterschiedliche Programme und Tools verwendet. Es folgt eine kurze Auflistung dieser:
\begin{itemize}
	\item \textbf{\LaTeX}: \LaTeX \ wird zur Erstellung dieses Dokuments verwendet. \LaTeX \ ist eine freie Software mit der Texte formatiert werden können und ist eine Alternative zu Textverarbeitungsprogrammen wie Microsoft Word. Insbesondere ist \LaTeX \ ein weit verbreitetes System zur Erstellung von wissenschaftlichen Dokumenten. \cite[vgl.][]{latex_o.J.}
	\item \textbf{Word} und andere Editoren: Word wird zur Erstellung firmeninterner Dokumente und Notizen für die spätere Dokumentation verwendet.
	\item \textbf{Excel}: Excel wird zur Erstellung der Bewertungsmatrix der Hardwareauswahl benötigt.
	\item \textbf{Onedrive}: Onedrive wird verwendet, um die Notizen und Dokumente zu speichern und im Projektteam untereinander zugänglich zu machen.
	\item \textbf{Github}: Mit Github wird der Programmcode verwaltet und gespeichert.
	\item \textbf{Discord und Whatsapp}: Discord und Whatsapp dienen der internen Kommunikation zwischen den Teammitgliedern.
	\item \textbf{Outlook und Teams}: Outlook und Teams werden für die Kommunikation mit der Firma und dem Betreuungslehrer verwendet.
	\item \textbf{Shortbus}: Bei Shortbus handelt es sich um ein Programm, mit dem über eine \acs{gui} \gls{modbus} Register ausgelesen und gesetzt werden können. Damit kann die \gls{modbus} Kommunikation getestet und die relevanten Register bestimmt werden. Siehe Kapitel \ref{shortbus_chapter} \nameref{shortbus_chapter}.
	\item \textbf{Visual Studio Code und PyCharm}: Mit diesen beiden integrierten Entwicklungsumgebungen (IDEs) wird das \gls{gls_python} Programm der \ac{rltanzeige} entwickelt.
	\item \textbf{Figma}: Figma wird zur Erstellung des Designs \bzw der \ac{gui} für die \ac{rltanzeige} benötigt. Siehe Kapitel \ref{figma_design} \nameref{figma_design}.
	\item \textbf{Zotero}: Bei Zotero handelt es sich um ein Literaturverwaltungsprogramm. Damit werden Internetquellen gesammelt, die später für den Import in \LaTeX \ exportiert werden können.
    \item \textbf{Raspberry PI Imager und balenaEtcher}: Mit diesen beiden Programmen können \gls{image}s auf die SD-Karte eines Raspberry PIs geflasht werden. Raspberry PI Imager wurde während der Entwicklung verwendet, während balenaEtcher beim Flashen des fertigen \gls{image}s verwendet wurde.
\end{itemize}
