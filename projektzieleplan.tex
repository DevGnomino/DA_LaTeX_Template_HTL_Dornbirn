In einem Projektzieleplan werden erwartete, messbare Ergebnisse beschrieben und zwischen den folgenden Punkten unterteilt:
\begin{itemize}
	\item \textbf{Haupt-Ziele:} An den Haupt-Zielen wird der  Projekterfolg gemessen. Sie umfassen die Ergebnisse der Diplomarbeit.
	\item \textbf{Neben-Ziele:} Neben- \bzw Zusatz-Ziele sind spezifische Ziele, die zusätzlich neben dem Hauptziel verfolgt werden. Sie können die Qualität eines Produktes erhöhen \bzw den Gesamterfolg des Produktes unterstützen.
	\item \textbf{Nicht-Ziele:} Nicht-Ziele dienen der genaueren Eingrenzung, was in einem Projekt erreicht werden soll. Damit werden Aktivitäten und Prozesse klar ausgegrenzt.
\end{itemize}
Die zu erreichenden Ziele werden meist innerhalb eines Meetings im Team ausgemacht. Treten Veränderungen bei diesen auf, durch beispielsweise Komplikationen, müssen diese Änderungen der Ziele dokumentiert werden \cite[vgl.][]{Diplomarbeiten-bbs:o.J.}. \\ 
Die folgende Tabelle \ref{tab:ziele_plan} beschreibt die Haupt-, Neben- und Nicht-Ziele, die für die Umsetzung dieser Diplomarbeit gelten.
\begin{table}[htpb]
	\caption{Zieleplan}
	\label{tab:ziele_plan}
	\begin{tabular}{p{\dimexpr 0.15\textwidth-2\tabcolsep} | p{0.80\textwidth}}
		\toprule
		\textbf{Zielart} & \textbf{Projektziele} \\
		\midrule
		& Visualisierung der Lüftungsgerät-Werte
		\\
		& Muss parametrierbar ausgeführt werden
		\\
		Hauptziele & Software-Schnittstelle (Zugriff mittels RS323 Schnittstelle)
		\\
		& Kosten auf wirtschaftlich sinnvoller Ebene
		\\
		& Gehäuse IP66 geschützt 
		\\
		\midrule
		& Nur Werte anzeigen, die auch vorhanden sind (Modbus)
		\\
		& Anleitung für User und Techniker erstellen
		\\
		Nebenziele & Die Anzeige wurde als Client definiert. Jetzt sollte sie als Server definiert werden
		\\
		& Soll QR-Code haben, der beim Scannen die Anzeige am Handy anzeigt
		\\
		\midrule
		& Soll eine Steuerungseinheit darstellen
		\\
		Nichtziele & Benutzerverwaltungsmöglichkeit
		\\
		& Bildschirm- und Bediensperre
		\\
		\bottomrule
	\end{tabular}
\end{table}