Zur Überprüfung, ob ein Projekt möglicherweise nicht zur richtigen Zeit beendet wird, kann der Meilensteinplan als Richtwert genommen werden. Er bildet die wichtigsten Ereignisse inklusive einer Deadline ab. So erhält man eine grobe Übersicht über Verzögerungen und es können Maßnahmen bei Bedarf ergriffen werden. Im Allgemeinen dient er aber auch der Mitarbeitermotivation bei erreichen eines Zwischenziels oder Meilensteins \cite[vgl.][]{domendos:2016}. \\
Anstehend in Tabelle \ref{tab:meilensteinplan} befinden sich die Meilensteine inklusive deren Termine.

\begin{table}[H]
	\caption{Projektmeilensteinplan}
	\label{tab:meilensteinplan}
	\begin{tabular}{p{\dimexpr 0.10\textwidth-2\tabcolsep} | p{0.30\textwidth} | p{0.15\textwidth} | p{0.15\textwidth} | p{0.15\textwidth}}
		\toprule
		\textbf{PSP-Code} & \textbf{Meilenstein} & \textbf{Basis-Termine} & \textbf{Aktuelle Plantermine} & \textbf{Ist Termine} \\
		\midrule
		1.1.1 & M1 Projekt ist gestartet & 09.05.2023 & 09.05.2023 & 09.05.2023 \\
		\midrule
		1.4.7 & M2 Anzeige funktioniert & 02.08.2023 & 03.08.2023 & 03.08.2023 \\
		\midrule
		1.4.10 & M3 Praktische Umsetzung abgeschlossen & 03.08.2023 & 04.08.2023 & 04.08.2023 \\
		\midrule
		1.5.4 & M4 Theoretische Umsetzung abgeschlossen & 01.03.2024 & 01.03.2024 & 01.03.2024 \\
		\midrule
		1.5.5 & M5 Diplomarbeit ist abgegeben & 03.04.2024 & 03.04.2024 & 03.04.2024 \\
		\midrule
		1.5.6 & M6 Abschlusspräsentation ist gehalten & 13.06.2024 & 13.06.2024 & 13.06.2024 \\
		\midrule
		1.6 & M8 Projekt ist abgeschlossen & 13.06.2024 & 13.06.2024 & 13.06.2024 \\
		\bottomrule
	\end{tabular}
\end{table}
