Für die Ausarbeitung der Diplomarbeit stehen eine Vielzahl von Programmiersprachen zur Verfügung. Geachtet wurde bei der Auswahl auf die Einfachheit bzw. Vertrautheit der einzelnen Teammitglieder mit der Sprache, deren Funktionalität, das Bibliotheken-Angebot und die vorhandene Dokumentation geachtet. In dieser Diplomarbeit wird die Programmiersprache Python verwendet. 

\subsection{Python}
Python ist eine vielseitig einsetzbare Programmiersprache. Eine erste Version wurde 1991 von Guido van Rossum als Hobbyprojekt entwickelt und später als sie erste Erfolge verzeichneten, wurde das Team erweitert. Seither konnte sich Python besonders durch seine einfache Syntax und die vielen vorhandenen Bibliotheken immer stärker etablieren. Im Gegensatz zu Sprachen wie C, C\#, etc. verwendet Python keinen Compiler, um das Coding vor der Ausführung des Programms in Maschinencode umzuwandeln, sondern einen Interpreter, der beim Start des Programms jede Zeile nacheinander überprüft und ausführt. Diese Praktik wurde erstmals mit der Programmiersprache Lisp eingeführt und wird unteranderem von Java, Ruby oder PHP benutzt. (Vielleicht hier noch auf Interpreter Vor- und Nachteile eingehen). 

\paragraph{Vorteile}
\begin{list}{}{}
	\item Nicht nur die Syntax ist übersichtlich, sondern auch die Bibliotheken-Verwaltung ist einfach. Mit dem pip-Paketmanager können mit einem „pip install“ alle vorhandenen Bibliotheken installiert werden.
	\item Irgendwas
\end{list}


\paragraph{Nachteile}
\begin{list}{}{}
	\item Da die Sprache Zeile für Zeile interpretiert bzw. ausgeführt wird, dauert die Ausführung länger. (faktisch belegen
	\item Irgendwas 2
\end{list}

