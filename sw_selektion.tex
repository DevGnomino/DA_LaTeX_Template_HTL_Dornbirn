Für die Ausarbeitung der Diplomarbeit stehen eine Vielzahl von Programmiersprachen zur Verfügung. Geachtet wurde bei der Auswahl auf die Einfachheit bzw. Vertrautheit der einzelnen Teammitglieder mit der Sprache, deren Funktionalität, das Bibliotheken-Angebot und die vorhandene Dokumentation. In dieser Diplomarbeit wird die Programmiersprache Python verwendet.
(LOGO VON PYTHON RECHTS OBEN HINKLATSCHEN) 

\subsection{Python}
Python ist eine vielseitig einsetzbare Programmiersprache. Eine erste Version wurde 1991 von Guido van Rossum als Hobbyprojekt entwickelt und später als sie erste Erfolge verzeichneten, wurde das Team erweitert. Seither konnte sich Python besonders durch seine einfache Syntax und die vielen vorhandenen Bibliotheken immer stärker etablieren. Im Gegensatz zu Sprachen wie C, C\#, etc. verwendet Python keinen Compiler, um das Coding vor der Ausführung des Programms in Maschinencode umzuwandeln, sondern einen Interpreter, der beim Start des Programms jede Zeile nacheinander überprüft und ausführt. Diese Praktik wurde erstmals mit der Programmiersprache Lisp eingeführt und wird unter anderen von Java, Ruby oder PHP benutzt. 
\cite{Python_Software_Foundation:o.J., Pramanick_gfg:2019, Ryte:2021}

Die folgenden Vor- und Nachteile beziehen sich auf \textcite{Ceaseo:2020}.
\paragraph{Vorteile}
\begin{itemize}
	\item Nicht nur die Syntax ist übersichtlich, sondern auch die Bibliotheken-Verwaltung ist einfach. Mit dem pip-Paketmanager können mit einem „pip install“ alle vorhandenen Bibliotheken installiert werden.
	\item Python kann in den unterschiedlichsten Anwendungen zum Einsatz kommen. Es können damit unteranderem Web-, Mobile- und Backendanwendungen erstellt werden. Außerdem eignet sie sich auch als Skriptsprache für Probleme, bei denen es keiner komplexen Software bedarf. Es können objektorientierte Strukturen genauso wie funktionale verwendet werden. 
\end{itemize}

\paragraph{Nachteile}
\begin{itemize}{}{}
	\item Da die Sprache Zeile für Zeile interpretiert wird, dauert die Ausführung länger als bei kompilierten Sprachen.
	\item Zur Initialisierung einer Variable muss kein Datentyp angegeben werden. Dadurch hat der Programmierer weniger Schreibaufwand, allerdings können in erhöhtem Maße Laufzeitfehler auftreten und auch die Übersichtlichkeit ist betroffen. Genauso ist die Abwesenheit von Klammern eine kontroverse Funktionalität.
\end{itemize}

\paragraph{Gründe für die Verwendung}
Ein ausschlaggebender Grund für die Verwendung von Python ist, dass alle Teammitglieder damit schon Erfahrungen gemacht haben. Beim Einsatz anderer Sprachen wäre die Einarbeitungszeit einzelner Teammitglieder zu berücksichtigen. Außerdem liegt es nahe, da es für den Raspberry PI eine Standardsprache ist. Die oben beschriebenen Nachteile der Sprache sind für ein Proof-of-Concept vernachlässigbar. Außerdem hat die Recherche ergeben, dass viele gut dokumentierte Bibliotheken für Modbus und \aclp{gui} vorhanden sind.

\paragraph{Verwendete Python-Bibliotheken}
Für das Auslesen der \acfp{rlt} Werte über das Modbus Protokoll wird minimalmodbus verwendet. Um die erhaltenen Werte auch grafisch anzuzeigen, wird customtkinter verwendet. Es folgen die Beschreibungen dieser beiden Bibliotheken.

