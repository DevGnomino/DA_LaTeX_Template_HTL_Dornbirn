Im Laufe eines jeden Projekts entstehen diverse Probleme und Hürden, die sich negativ auf die Geschwindigkeit, Motivation und Effizienz auswirken können und somit mögliche Meilensteine und Zeitpläne gefährden. Nichtsdestotrotz sind sie ein unentbehrlicher Bestandteil des Prozesses bzw. der Entwicklung, da sie einen dazu zwingen, nach alternativen Lösungsmöglichkeiten zu suchen, wodurch die Teamarbeit gestärkt wird und man sich als Person neue Fähigkeiten und neues Wissen aneignet.\\ Folgend sind die größten Probleme festgehalten, die während dieser Diplomarbeit auftraten:

\begin{enumerate}
	\item \textbf{Hardware: }Nachdem die Hardwareevaluierung abgeschlossen und die Entscheidung dem Projektleiter Simon Köldorfer mitgeteilt worden war, wurde eine Anzahl an Raspberry Pis, SD-Cards und ein Display zur Verfügung gestellt, um bereits während des Wartens auf die eigentlichen Komponenten mit teilweise erhöhten Lieferzeiten bereits arbeiten zu können. Relativ schnell wurde festgestellt, dass auf der einen SD-Card nur eine veraltete Version von Python installierbar war und auf der anderen die Python-Version zwar auf die derzeit neueste aktualisiert werden konnte, aber das Display aus unbekannten Gründen nicht damit kompatibel war. Da nun die SD-Card mit der veralteten Version ausgeschlossen wurde, musste eine Lösung für die andere gefunden werden. Mit der Applikation \enquote{Real VNC Viewer} konnte, ohne Verwendung des Displays, auf den Raspberry Pi zugegriffen werden. So wurde mit dem Programm als Anzeigelösung gearbeitet, bis das eigentlich gewünschte Display die Firma erreicht und auch funktionierende SD-Cards zur Verfügung standen. Nach wenigen Tagen wurde dann ein weiteres Problem bemerkt, denn der Raspberry Pi konnte übergangslos keine WLAN-Verbindung mehr herstellen. Dies zwang dazu, auf eine außergewöhnliche Methode umzusteigen, damit der Zugriff auf den Raspberry Pi gewährt werden konnte. Durch die Verwendung mehrerer Ethernetkabel, die mit dem Mikrocontroller, den Laptops und einem Switch verbunden wurden, konnte der Zugriff mittels des Programmes \enquote{PuTTy} gewährt werden.
	\item \textbf{Lieferung der Komponenten: } Bevor die gewünschte und erwartete Hardware in der Firma eintraf, wurde der Code bereits passend für das eigentlich bestellte Display geschrieben. Aufgrund von Komplikationen konnte besagtes Gerät nicht beschaffen werden, was dazu führte, dass der Code vollständig überarbeitet werden musste. Zudem wurde das neue Display über \gls{hdmi} an den Raspberry Pi angeschlossen und nicht über \gls{gpio}, wie es bei dem gewünschten Display gewesen wäre. Diese unerwarteten Veränderungen haben die Entwicklungszeit verlängert und erforderten zusätzliche Anpassungen.
	\item \textbf{Modbus-Auslesung: } Ein weiteres Problem wurde bei der Auslesung der Modbus-Daten festgestellt. Obwohl das Lesen der Temperatur- und Drucksensoren schnell und akkurat sowohl in \enquote{Shortbus} als auch im Code funktionierte, wurde ein Hindernis bei den Ventilatoren angetroffen, da hier hexadezimale Zahlen bei den Modbusregistern verwendet wurden und somit auch im eben genannten Programm einige Einstellungen geändert werden mussten. Nach langer Suche nach den richtigen Einstellungen konnte nun endlich ein weiterer Erfolg verzeichnet werden. Als dann der Code am Lüftungsgerät ausprobiert wurde, stieß man auf ein weiteres Problem in dieser Thematik. Im Code befand sich fälschlicherweise eine Zeile, die zwar passend für Druck- und Temperatursensoren war, aber nicht für die Ventilatoren. Bevor diese Zeile gefunden worden ist und der Code sowie die Datenauslesung einwandfrei funktionieren konnten, wurde die Suche vergebens nach neuen Modbus-Libraries gestartet.
	\item \textbf{Modbus-Adapter und Gehäuse: }
	\item \textbf{Kommunikation: }
\end{enumerate}
