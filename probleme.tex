Eine Normalität während der Ausarbeitung eines Projektes sind auftretende Probleme. Während kleinere den Arbeitsfluss nicht stören, können größere wiederum Einbußen in den Zeitplänen bedeuten. Nichtsdestotrotz stärken sie im Allgemeinen das Teamwork und man entwickelt sich auf persönlicher Ebene.\\ Folgend sind die größten Probleme festgehalten, die während dieser Diplomarbeit auftraten.

\begin{enumerate}
	\item \textbf{Hardware:} Nachdem die Hardwareevaluierung abgeschlossen und die Entscheidung dem Projektbetreuer Simon Köldorfer mitgeteilt wurde, wurde dem Team eine Anzahl an Raspberry PIs, SD-Karten und ein Display zur Verfügung gestellt, um bereits während des Wartens auf die eigentlichen Komponenten, mit teilweise erhöhten Lieferzeiten, bereits arbeiten zu können. Relativ schnell wurde festgestellt, dass auf der einen SD-Karte nur eine veraltete Version von \gls{gls_python} installierbar war und auf der anderen die \gls{gls_python}-Version zwar auf die derzeit neueste aktualisiert werden konnte, aber das Display aus unbekannten Gründen nicht damit kompatibel war. Da nun die SD-Karte mit der veralteten Version ausgeschlossen wurde, musste eine Lösung für die andere gefunden werden. Mit der Applikation \enquote{Real \gls{gls_vnc} Viewer} konnte, ohne Verwendung des Displays, auf den Raspberry PI zugegriffen werden. So wurde mit dem Programm als Anzeigelösung gearbeitet, bis das eigentlich gewünschte Display die Firma erreicht und auch funktionierende SD-Karten zur Verfügung standen. Nach wenigen Tagen wurde dann ein weiteres Problem bemerkt, denn der Raspberry PI konnte übergangslos keine WLAN-Verbindung mehr herstellen. Dies zwang dazu, auf eine außergewöhnliche Methode umzusteigen, damit der Zugriff auf den Raspberry PI gewährt werden konnte. Durch die Verwendung mehrerer Ethernetkabel, die mit dem Mikrocontroller, den Laptops und einem Switch verbunden wurden, konnte der Zugriff mittels des Programmes \enquote{PuTTy} gewährt werden.
	\item \textbf{Lieferung der Komponenten:} Bevor die gewünschte und erwartete Hardware in der Firma eintraf, wurde der Code bereits passend für das eigentlich bestellte Display geschrieben. Aufgrund von Komplikationen konnte besagtes Gerät nicht beschafft werden, was dazu führte, dass der Code vollständig überarbeitet werden musste. Zudem wurde das neue Display über \ac{hdmi} an den Raspberry PI angeschlossen und nicht über \ac{gpio}, wie es bei dem gewünschten Display gewesen wäre. Diese unerwarteten Veränderungen haben die Entwicklungszeit verlängert und erforderten zusätzliche Anpassungen.
	\item \textbf{\gls{modbus}-Auslesung:} Ein weiteres Problem wurde bei der Auslesung der \gls{modbus}-Daten festgestellt. Obwohl das Lesen der Temperatur- und Drucksensoren schnell und akkurat sowohl in \enquote{Shortbus} als auch im Code funktionierte, wurde ein Hindernis bei den Ventilatoren angetroffen, da hier hexadezimale Zahlen bei den Modbusregistern verwendet wurden und somit auch im eben genannten Programm einige Einstellungen geändert werden mussten. Nach langer Suche nach den richtigen Einstellungen konnte nun endlich ein weiterer Erfolg verzeichnet werden. Als dann der Code am Lüftungsgerät ausprobiert wurde, stieß man auf ein weiteres Problem in dieser Thematik. Im Code befand sich fälschlicherweise eine Zeile, die zwar passend für Druck- und Temperatursensoren war, aber nicht für die Ventilatoren. Bevor diese Zeile gefunden wurde und der Code sowie die Datenauslesung einwandfrei funktionieren konnten, wurde die Suche vergebens nach neuen \gls{modbus}-Bibliotheken gestartet.
	\item \textbf{\gls{modbus}-Adapter und Gehäuse:} Im Zusammenhang mit der Hardware traten zwei zusätzliche Probleme auf, die den Abschluss des praktischen Teils verzögerten. Aufgrund von Lieferschwierigkeiten mit dem Display wurde eine Alternative beschafft, die jedoch ebenfalls Komplikationen mit sich brachte, da sie seitlich statt unten angeschlossen wurde. Dies führte dazu, dass das zuvor ausgewählte Gehäuse zu klein war. Die Firma Bösch entschied sich, das richtige Gehäuse selbst per 3D-Drucker herzustellen. Zudem gab es eine Herausforderung mit dem \gls{gls_rs485}-Adapter für den Raspberry PI, welche erst bei der Rückkehr in die Firma nach Beendigung des Praktikums gelöst werden konnte, indem andere Adapter dieser Art getestet wurden.
	\item \textbf{Kommunikation:} Dadurch, dass das Pflichtenheft nicht direkt von Anfang an sorgfältig bearbeitet und unterschrieben wurde, konnte nicht verhindert werden, dass in internen Meetings weitere Spezifikationen und Anforderungen an das Projekt hinzugefügt wurden, welche zusätzlichen Aufwand und Verzögerungen mit sich brachten. Bezüglich Meetings hätte auch das Design-Meeting viel früher stattfinden müssen, denn es wurde erst im späteren Projektverlauf klar, welche Werte tatsächlich auf dem Display angezeigt werden sollten. Beide genannten Punkte hätten vermieden werden könne, wenn die Kommunikation mit dem Projektbetreuer aktiver gewesen wäre.
\end{enumerate}
