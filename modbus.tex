Der Modbus Standard definiert ein Application Layer Kommunikations Protokoll, dass sich auf Schicht 7 des OSI-Modells befindet. Es bietet Client/Server Kommunikation zwischen Geräten, die auf verschiedenen Bussen oder Netzwerken angeschlossen sind. (direktes Zitat; offizielle Definition)
Modbus verwendet das Client/Server System. In einem Bussystem kann es nur einen Server geben. Es können jedoch beliebig viele Clients am Bus angeschlossen werden. Der Server kann die Kommunikation mit den einzelnen Clients initialisieren. Er kann ihnen Daten senden, aber auch von ihnen Daten anfordern. Ein Client hingegen kann keine Kommunikation beginnen.
Es gibt zwei verschiedene Einteilungen des Modbus Protokolls: 
\begin{itemize}
\item \textbf{Modbus Application Protocol:} Befindet sich auf der siebten Schicht des OSI-Modells. Dabei können die Geräte an einem Bus oder an einem Netzwerk angeschlossen werden.
\item \textbf{Modbus Serial Line Protocol:} Befindet sich auf der zweiten Schicht des OSI-Modells. Die Geräte sind hier an einem seriellen Bus angeschlossen.
\end{itemize}
\cite{Modbus_Organization_AP:2012}
\cite{Modbus_Organization_SL:2012}

\subsection{Entstehung des Modbus-Protokolls}

\subsection{Vor- und Nachteile}


\subsection{Funktionsweise}


\subsection{Ausführungsarten (RTU, TCP, ASCII)}


\subsection{Serielle Kommunikation (bei RTU/RS485)}


