Die offizielle Definition des Modbus Protokolls bezogen von der Modbus Organization \cite{Modbus_Organization_AP:2012} lautet:
\begin{quotation}
	\emph{
		MODBUS is an application layer messaging protocol, positioned at level 7 of the OSI model, which provides client/server communication between devices connected on different types of buses or networks.}
\end{quotation}

Der Modbus Standard definiert ein Application Layer Kommunikations Protokoll, dass sich auf Schicht 7 des OSI-Modells befindet. Es bietet Client/Server Kommunikation zwischen Geräten, die auf verschiedenen Bussen oder Netzwerken angeschlossen sind.

Entstehung einfach hier reinquetschen, weils nicht so viel ist

Es gibt zwei verschiedene Einteilungen des Modbus Protokolls: 
\begin{itemize}
\item \textbf{Modbus Application Protocol:} Befindet sich auf der siebten Schicht des OSI-Modells. Dabei können die Geräte an einem Bus oder an einem Netzwerk angeschlossen werden. Genauere Information über die Ausführungsart mit dem TCP Protokoll sind später in diesem Kapitel zu finden.
\item \textbf{Modbus Serial Line Protocol:} Befindet sich auf der zweiten Schicht des OSI-Modells. Die Geräte sind hier an einem seriellen Bus angeschlossen. Dabei gibt es hier zwei Ausführungsarten, nämlich RTU und ASCII. Diese werden später ausführlicher beschrieben.
\end{itemize}
\cite{Modbus_Organization_AP:2012}
\cite{Modbus_Organization_SL:2012}

\subsection{Funktionsweise}
Modbus verwendet das Client/Server System. In einem Bussystem kann es nur einen Server geben. Es können jedoch beliebig viele Clients am Bus angeschlossen werden. Der Server kann die Kommunikation mit den einzelnen Clients initialisieren. Er kann ihnen Daten senden und von ihnen Daten anfordern. Ein Client hingegen kann keine Kommunikation beginnen.

\subsection{Vor- und Nachteile}


\subsection{Ausführungsarten}
\paragraph{RTU}
Funktioniert über die serielle Schnittstelle.
\paragraph{ASCII}
\paragraph{TCP}

\subsection{Serielle Kommunikation (bei RTU/RS485)}


