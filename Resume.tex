\ifoot{\leftmark}
%gemeinsames Kapitel
\chapter{Zusammenfassung}
\noindent Das Hauptziel war es eine kostengünstige und konfigurierbare Anzeige zu erstellen. Dieses Ziel wurde im Rahmen dieser Diplomarbeit erfüllt. Es wurde ein Prototyp erstellt, dessen Hardwarekomponenten nach einer kosten-optimierten Evaluierung ausgewählt und in einem Gehäuse verbaut wurden. Nicht nur die Hardware, sondern auch die Software wurde nach den Vorgaben erstellt. Somit ist die Anwendung mittels \acs{json} Konfigurationsdateien parametrierbar und übernimmt keine Steuerungsaufgaben. Die ausgewählten Technologien erwiesen sich als zufriedenstellend und harmonierten gut miteinander. Allerdings wurde beim Beschreiben des Codings und der Konfigurationsdateien erkannt, dass der Code etwas unübersichtlich und unstrukturiert ist. Die Erkenntnis daraus ist, dass man sich im Vorhinein genauer mit dem Aufbau des Programms beschäftigen sollte.

Das Nebenziel, also die Einrichtung einer weiteren Modbus Schnittstelle, wurde nicht erreicht, da dafür keine Zeit mehr blieb. Diese ist allerdings essenziell, damit die Anzeige auch an den Lüftungsgeräten verbaut werden kann. Dies ist nötig, da ein Modbus System nur einen Client haben kann und sowohl die \acs{rltanlage} als auch die zentrale Steuerung des Lüftungsgerätes Client sein müssen.

Die interne Teamarbeit verlief zufriedenstellend, jedoch gab es Schwierigkeiten in der Kommunikation mit der Firma und dem Betreuer. Die genaue Definition des Ziels erfolgte erst spät, was dazu führte, dass das entwickelte Programm umgebaut werden musste. Vorteilhaft wäre eine genauere Festlegung der Ziele im Pflichtenheft und eine aktivere Kommunikation mit den beteiligten Personen.

Empfehlenswert ist es den Programmcode einer Verbesserung zu unterziehen und auch das Konfigurationsdateikonzept zu vereinfachen, da dieses etwas komplex ist. Außerdem ist es wichtig, dass eine weitere Modbus Schnittstelle eingerichtet wird.
Es folgt eine Präsentation bei der Geschäftsführung der Walter Bösch GmbH, in der auch das weitere Vorgehen besprochen wird.