\subsection{Einlesen der Konfigurationsdateien}
!Klassenstruktur beschreiben + UML-Diagramm; den ganzen code dazu weglassen --> weil sehr viel - UML struktur im GUI schon beschrieben; also hier kurz halten; vlt gui section vor config dateien verschieben!

Im Backend (falls man es so nennen kann) wird die Klassenstruktur der \acs{gui} nachgebildet. Es gibt eine Page-, Measurement- und Sensorklasse. Diese existieren, damit der Text an der richtigen Stelle verändert werden kann. Eine genauere Beschreibung findet sich hier (Grafische Benutzeroberfläche). 

!Config eingelesen!

\begin{pythoncode}
class Measurement():
	def __init__(self, description, unit, sensors, python_function, additional_info):
	self.description = description
	self.unit = unit
	self.sensors = sensors
	self.python_function = python_function
	self.additional_info = additional_info
	self.value = "N/A"
\end{pythoncode}

\begin{pythoncode}
	def get_sensor_data(device_full_data, port_name, sensor_unit):
	for device in device_full_data["ports"]:
	print(device)
	if device["port"] == port_name:
	print("in port if")
	sensor_register = device["register"]
	sensor_function_code = device["function_code"]
	if "units" in device:
	for unit_pair in device["units"]:  # Hier kann vielleicht später das Array anders entpackt werden (mit *)
	if unit_pair["unit"] == sensor_unit:
	sensor_scaling = unit_pair["scaling"]
	return_var = {"sensor_register": sensor_register, "sensor_scaling": sensor_scaling,
		"sensor_function_code": sensor_function_code}
	return return_var
	else:
	return_var = {"sensor_register": sensor_register, "sensor_scaling": 1,
		"sensor_function_code": sensor_function_code}
	return return_var
	
	
	return -1
	
	
	def get_sensor_unit(sensor_name):
	sensor_file = open(DEVICES_CONFIG_PATH + 'sensors.json', encoding='utf-8')
	sensor_full_data = json.load(sensor_file)
	# print(sensor_full_data)
	for sensor in sensor_full_data:
	if sensor["type"] == sensor_name:
	sensor_unit = sensor["unit"]
	return sensor_unit
	
	sensor_file.close()
	return ""
\end{pythoncode}

!Im entsprechenden Sensor gespeichert!