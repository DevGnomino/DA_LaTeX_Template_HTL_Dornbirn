!Klassenstruktur beschreiben + UML-Diagramm; den ganzen code dazu weglassen --> weil sehr viel - UML struktur im GUI schon beschrieben; also hier kurz halten; vlt gui section vor config dateien verschieben!

Im Backend (falls man es so nennen kann) wird die Klassenstruktur der \acs{gui} nachgebildet. Es gibt eine Page-, Measurement- und Sensorklasse. Diese existieren, damit der Text an der richtigen Stelle verändert werden kann. Eine genauere Beschreibung findet sich hier (Grafische Benutzeroberfläche). 

!Config eingelesen!

!Im entsprechenden Sensor gespeichert!