\chapter{Aufgabenstellung} 
\label{aufgabenstellung}
Im Rahmen der Diplomarbeit soll eine zentrale Visualisierung entwickelt werden, die alle 
wichtigen Informationen des Lüftungsgerätes sammelt und direkt am Gerät anzeigt. Unter 
den anzuzeigenden Informationen befinden sich alle Temperaturwerte, Druckwerte und 
Leistungsdaten, wobei bei Letzteren die Effizienz der RLT-Anlage berechnet wird. So können 
Fehler der Anlage schnell identifiziert werden.
Die Anzeige soll parametrierbar ausgeführt werden, um für die unterschiedlichsten 
Lüftungsgeräte verwendet werden zu können. Dafür können gegebenenfalls Funktionen von 
Modbus verwendet werden, womit automatisch nachgeschaut wird, welche Werte empfangen 
werden und diese infolge angezeigt werden. Die Grunddaten, wie z.B. welche Art von 
Sensor bzw. an welcher Stelle (z.B. Analog Input 1) dieser angeschlossen ist, werden über 
ein JSON-Config-File eingelesen. Dieses File wird von dem/der Techniker*in, die den 
Raspberry dann aufsetzt, ausgefüllt.
Da die Anzeige an den meisten Lüftungsanlagen verbaut wird, sollten sich die Kosten für 
diese in einem wirtschaftlich sinnvollen Rahmen bewegen.
Die Anzeige soll NICHT für Steuerungsaufgaben verwendet werden. Benutzerverwaltung, 
Bildschirm- und Bediensperren sind ebenfalls nicht vorzusehen. \\

Wenn es gelingt, den Modbus auszulesen und diesen grafisch darzustellen (Raspberry-Pi ist 
Master), dann soll eine zweite Modbus Schnittstelle einrichten. Diese zweite 
Schnittstelle ist als Slave auszuführen und soll alle, auf der ersten Modbus Linie
gesammelte Daten bereitstellen. Dies ermöglicht eine einfache Anbindung von fremd 
Steuerungen ohne Gefahr zu laufen, dass von dieser Parameter im Lüftungsgerät via 
Modbus verstellt werden. Voraussetzung dafür ist jedoch, dass alles, was zuvor besprochen
wurde, umgesetzt ist.