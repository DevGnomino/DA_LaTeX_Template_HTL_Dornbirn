\chapter{Einleitung} 
\label{aufgabenstellung}
\noindent Im Rahmen der Diplomarbeit soll eine zentrale Visualisierung entwickelt werden, die alle wichtigen Informationen der \ac{rltanlage} sammelt und direkt am Gerät anzeigt. Unter den anzuzeigenden Informationen befinden sich jegliche Temperaturwerte, Druckwerte und Leistungsdaten, wobei bei letzteren die Effizienz der \ac{rltanlage} berechnet wird. So können Fehler der Anlage schnell identifiziert werden.
Die Anzeige soll parametrierbar ausgeführt werden, um für die unterschiedlichsten 
\acp{rltanlage} verwendet werden zu können. Die Grunddaten, wie \zB welche Art von 
Sensor \bzw an welcher Stelle dieser angeschlossen ist, werden aus \ac{json} Konfigurationsdateien ausgelesen. Diese Dateien werden von der Technikerin oder dem Techniker, die oder der den Raspberry PI dann aufsetzt, ausgefüllt.
Da die \ac{rltanzeige} an den meisten \acp{rltanlage} verbaut werden soll, sollten sich die Kosten für diese in einem wirtschaftlich sinnvollen Rahmen bewegen.
Die \ac{rltanzeige} soll nicht für Steuerungsaufgaben verwendet werden. Benutzerverwaltung, 
Bildschirm- und Bediensperren sind ebenfalls nicht vorzusehen. \\

Nebenziel:
Wenn es gelingt, die Messwerte über \gls{modbus} auszulesen und diese grafisch darzustellen (Raspberry PI ist dabei ein \gls{modbus} Client), dann soll eine zweite \gls{modbus} Schnittstelle eingerichtet werden. Diese zweite Schnittstelle ist als Server auszuführen und soll alle auf dem ersten \gls{modbus} Bus gesammelte Daten bereitstellen. Dies ermöglicht eine einfache Anbindung von fremden Steuerungen ohne Gefahr zu laufen, dass von dieser Parameter in der \ac{rltanlage} via \gls{modbus} verstellt werden. Voraussetzung dafür ist jedoch, dass das Hauptziel erreicht wurde.

%Kurzbeschreibung des Themas
%eschreibung der Leistung: Ziel der Arbeit? Für wen hat die Arbeit relevant? ...
%Darstellung der Vorgehensweise: In welche Kapitel ist die Arbeit aufgebaut? Was steht darin?