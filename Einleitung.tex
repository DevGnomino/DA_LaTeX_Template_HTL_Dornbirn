\chapter{Einleitung} 
\label{aufgabenstellung}
\noindent Im Rahmen der Diplomarbeit soll eine zentrale Visualisierung entwickelt werden, die alle 
wichtigen Informationen des Lüftungsgerätes sammelt und direkt am Gerät anzeigt. Unter 
den anzuzeigenden Informationen befinden sich alle Temperaturwerte, Druckwerte und 
Leistungsdaten, wobei bei Letzteren die Effizienz der RLT-Anlage berechnet wird. So können 
Fehler der Anlage schnell identifiziert werden.
Die Anzeige soll parametrierbar ausgeführt werden, um für die unterschiedlichsten 
Lüftungsgeräte verwendet werden zu können. Die Grunddaten, wie z.B. welche Art von 
Sensor bzw. an welcher Stelle (z.B. Analog Input 1) dieser angeschlossen ist, werden aus JSON Konfigurationsdateien ausgelesen. Diese Dateien werden von dem/der Techniker*in, die den 
Raspberry dann aufsetzt, ausgefüllt.
Da die Anzeige an den meisten Lüftungsanlagen verbaut wird, sollten sich die Kosten für 
diese in einem wirtschaftlich sinnvollen Rahmen bewegen.
Die Anzeige soll NICHT für Steuerungsaufgaben verwendet werden. Benutzerverwaltung, 
Bildschirm- und Bediensperren sind ebenfalls nicht vorzusehen. \\

Nebenziel:
Wenn es gelingt, die Messwerte über Modbus auszulesen und diese grafisch darzustellen (Raspberry Pi ist 
Client), dann soll eine zweite Modbus Schnittstelle eingerichtet werden. Diese zweite 
Schnittstelle ist als Server auszuführen und soll alle, auf dem ersten Modbus Bus
gesammelte Daten bereitstellen. Dies ermöglicht eine einfache Anbindung von fremd 
Steuerungen ohne Gefahr zu laufen, dass von dieser Parameter im Lüftungsgerät via 
Modbus verstellt werden. Voraussetzung dafür ist jedoch, dass das Hauptziel erreicht wurde.

%Kurzbeschreibung des Themas
%eschreibung der Leistung: Ziel der Arbeit? Für wen hat die Arbeit relevant? ...
%Darstellung der Vorgehensweise: In welche Kapitel ist die Arbeit aufgebaut? Was steht darin?