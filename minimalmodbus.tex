\subsection{Minimalmodbus}
Es handelt sich bei minimalmodbus um eine Open-Source Bibliothek. Die Bibliothek ermöglicht die Kommunikation verschiedener Geräte über das \gls{modbus} Protokoll mit \gls{gls_python}. Das Gerät, auf dem das \gls{gls_python} Programm läuft, muss der Server sein. Das Programm kann dann die einzelnen Clients ansprechen und ihre Register auslesen bzw. überschreiben.

Sie wurde hauptsächlich von Jonas Berg entwickelt, aber auch zahlreiche andere Entwickler haben dazu beigetragen. Sie ist in \gls{gls_python} geschrieben und ist eine Weiterentwicklung der pySerial-Bibliothek. Diese Abhängigkeit wird benötigt, um auf die seriellen Ports des Geräts zuzugreifen \cite{Liechti_pySerial:o.J.}. 

Die Bibliothek bietet \gls{modbus} \acs{rtu} und \gls{modbus} \acs{ascii} Funktionalitäten. In den Lüftungsgeräten der Firma Bösch wird über \gls{modbus} \acs{rtu} kommuniziert, daher ist diese Bibliothek einsetzbar. Außerdem bietet sie eine umfangreiche Dokumentation.
\cite{Berg_MiniModbus:2023, Berg_MiniModbus_Git:2023} 
%https://pypi.org/project/minimalmodbus/#description %gleiche infos wie auf github-repo und doku

Andere in Erwägung gezogene Bibliotheken für die \gls{modbus} Kommunikation sind PyModbus und Modbus-TK. Obwohl die Bibliothek von der Funktionalität sehr umfangreich ist, stellt sich die Einarbeitung in PyModbus als kompliziert heraus. Bei Modbus-TK hingegen reicht schon eine kurze Recherche nach Dokumentationsmaterial aus, um festzustellen, dass diese gar keine besitzt und damit unqualifiziert ist.