\subsection{Minimalmodbus}
Es handelt sich bei minimalmodbus um eine Open-Source Bibliothek. Die Bibliothek ermöglicht die Kommunikation verschiedener Geräte über das Modbus Protokoll mit Python. Das Gerät, auf dem das Python-Programm läuft, muss der Server sein. Das Programm kann dann die einzelnen Clients ansprechen und ihre Register auslesen bzw. überschreiben.

Sie wurde hauptsächlich von Jonas Berg entwickelt, aber auch zahlreiche andere Entwickler haben dazu beigetragen. Sie ist in Python geschrieben und ist eine Weiterentwicklung der pySerial-Bibliothek. Diese Abhängigkeit wird benötigt, um auf die seriellen Ports des Geräts zuzugreifen (https://pyserial.readthedocs.io/en/latest/pyserial.html). 

Die Bibliothek bietet Modbus RTU und Modbus ASCII Funktionalitäten. In den Lüftungsgeräten der Firma Bösch wird über Modbus RTU kommuniziert, daher ist diese Bibliothek einsetzbar. Außerdem bietet sie eine umfangreiche Dokumentation (https://minimalmodbus.readthedocs.io/en/stable/). 
%https://pypi.org/project/minimalmodbus/#description 

Andere in Erwägung gezogene Bibliotheken für die Modbus Kommunikation sind PyModbus und Modbus-TK. Obwohl die Bibliothek von der Funktionalität sehr umfangreich ist, stellt sich die Einarbeitung in PyModbus als kompliziert heraus. Bei Modbus-TK hingegen reicht schon eine kurze Recherche nach Dokumentationsmaterial aus, um festzustellen, dass diese gar keine besitzt und damit unqualifiziert ist.
