Um die Zuständigkeiten in Projekten regeln zu können, existieren Projektfunktionendiagramme. Diese werden in Form einer Tabelle dargestellt und beinhalten alle Projektbeteiligten sowie alle Arbeitspakete. Bei den Beteiligten gelten folgende Kürzel:
\begin{itemize}
	\item \textbf{PAG} - Projektauftraggeber
	\item \textbf{PL} - Projektleiter
	\item \textbf{PTM} - Projektteammitglied
	\item \textbf{PM} - Projektmitarbeiter
\end{itemize}
Unter jedem Beteiligten sind die jeweiligen Funktionen dessen eintragbar, die bei der Ausführung des Arbeitspakets eingenommen wurde.
Bei den Funktionen gelten folgende Kürzel:
\begin{itemize}
	\item \textbf{D} - Durchführungsverantwortung
	\item \textbf{M} - Mitarbeit
	\item \textbf{I} - bekommt Informationen
\end{itemize}
Hier ist zu beachten, dass die Verantwortung eines Arbeitspakets nur durch eine Person übernommen werden kann \cite{prezi:o.J.}.