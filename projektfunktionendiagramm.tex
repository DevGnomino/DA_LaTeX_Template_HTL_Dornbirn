Um die Zuständigkeiten in Projekten regeln zu können, existieren Projektfunktionendiagramme. Diese werden in Form einer Tabelle dargestellt und beinhalten alle Projektbeteiligten sowie alle Arbeitspakete. Bei den Beteiligten gelten folgende Kürzel:
\begin{itemize}
	\item \textbf{PAG} - Projektauftraggeber
        \item \textbf{PB} - Projektbetreuer
	\item \textbf{PL} - Projektleiter
	\item \textbf{PTM} - Projektteammitglied
	\item \textbf{PM} - Projektmitarbeiter
\end{itemize}
Unter den beteiligten Personen können die jeweiligen Zuständigkeiten eingetragen werden, die sie bei der Durchführung des Arbeitspakets übernommen haben.
Bei den Funktionen gelten folgende Kürzel:
\begin{itemize}
	\item \textbf{D} - Durchführungsverantwortung
	\item \textbf{M} - Mitarbeit
	\item \textbf{I} - bekommt Informationen
\end{itemize}
Hier ist zu beachten, dass die Verantwortung eines Arbeitspakets nur durch eine Person übernommen werden kann \cite[vgl.][]{prezi:o.J.}. \\
Für die folgende Tab. \ref{tab:funktionendiagramm}, die die Zuständigkeiten angibt, gelten folgende Kürzel: \\ 
\textbf{PAG}: Walter Bösch GmbH \& Co. KG \\\textbf{PB}: Simon Köldorfer \\\textbf{PTL}: \mangeng \\\textbf{PTM1}: \pezze \\\textbf{PTM2}: \fenkart \\\textbf{PTM3}: \schneider

\definecolor{mygray}{gray}{.7}
\definecolor{mygray2}{gray}{.9}
\newpage
\begin{longtable}{p{\dimexpr 0.10\textwidth-2\tabcolsep} p{0.25\textwidth} p{0.05\textwidth} p{0.05\textwidth}  p{0.10\textwidth} p{0.10\textwidth} p{0.10\textwidth} p{0.05\textwidth}}
	\caption{Projektfunktionendiagramm}
	\label{tab:funktionendiagramm}
	\\ \toprule
	\textbf{PSP-Code} & \textbf{AP-Bezeichnung} & & \textbf{PB} & \textbf{PL} & \textbf{PTM1} & \textbf{PTM2} & \textbf{PTM3}
	\\ \midrule
	\endfirsthead
	\caption{Projektfunktionendiagramm (Fortsetzung)}
	\\ \toprule
	\textbf{PSP-Code} & \textbf{AP-Bezeichnung} & & \textbf{PB} & \textbf{PL} & \textbf{PTM1} & \textbf{PTM2} & \textbf{PTM3}
	\\ \midrule
	\endhead
	%
	\midrule
	\multicolumn{7}{r}{{Fortsetzung auf  der nächsten Seite}} 
	\\ \bottomrule
	\endfoot
	%
	\bottomrule
	\endlastfoot
	\rowcolor{mygray} 1 & Diplomarbeit & & & & & & \\ \midrule
	\rowcolor{mygray2}1.1 & Projektmanagement & & & & & &\\ \midrule
	1.1.2 & Projekt gestartet & & & D & M & M & M \\ \midrule
	1.1.3 & Projektkoordination & & & D & & & \\ \midrule
	1.1.4 & Projektcontrolling & & & D & M & M & M \\ \midrule
	1.1.5 & Projektkontrolle & & & D & M & M & M \\ \midrule
	\rowcolor{mygray2}1.2 & Vorprojektzeit & & & & & & \\ \midrule
	1.2.1 & Suche nach Partnerfirma & & & D & & & \\ \midrule
	1.2.2 & Thematscheidung & & & D & M & M & M \\ \midrule
	1.2.3 & Festlegung des Themas & & & M & M & D & M \\ \midrule
	1.2.4 & Festlegung des Betreuungslehrers & & & M & D & M & M \\ \midrule
	\rowcolor{mygray2}1.3 & Planung & & & & & & \\ \midrule
	1.3.1 & Einführung in Lüftungsgeräte & & D & I & I & I & I \\ \midrule
	1.3.2 & Pflichtenheft überarbeiten & & & & M & & D \\ \midrule
	1.3.3 & Hardware evaulierung & & & I & I & D & I \\
	1.3.4 & Präsentation der Recherchen & & & M & D & M & M \\ \midrule
	\rowcolor{mygray2}1.4 & Projektbearbeitung & & & & & & \\ \midrule
	1.4.1 & Projekthandbuch & & & D & & &  \\ \midrule
	1.4.2 & Hardware zusammenbauen & & & M & M & M & D \\ \midrule
	1.4.3 & Hardware testen & & & M & M & D & M \\ \midrule
	1.4.4 & Code erstellen & & & & M & & D \\ \midrule
	1.4.5 & Config File erstellen & & & & D & & M \\ \midrule
	1.4.6 & Funktionalität testen & & & M & D & M  & M \\ \midrule
	1.4.8 & Bedienungsanleitung End User & & & D & & M & \\ \midrule
	1.4.9 & Service Anleitung Techniker & & I & M & M & D & M \\ \midrule
	\rowcolor{mygray2}1.5 & Abgabe und Doku & & & & & & \\ \midrule
	1.5.1 & Abgabe von erarbeitetem an Firma & & I & D & M & M & M \\ \midrule
	1.5.2 & Theoretischer Teil & & & M & M & M & D \\ \midrule
	1.5.3 & Zwischenpräsentation Schule & & & M & D & M & M \\ 
\end{longtable}