\newpage
\subsection{Selektion} \label{selektion}
Die im Kapitel \ref{evaluierung} auf der Grundlage der Recherche erstellten Varianten mussten während eines Meetings dem Projektleiter und einer externen Person präsentiert werden. Ziel war es, die Varianten zu analysieren und festzustellen, welche für die praktische Umsetzung sinnvoll sind und auch wirtschaftlich vertretbar erscheinen. Im Folgenden befindet sich eine kurze Beschreibung der jeweiligen Varianten in Kombination mit den im Meeting angemerkten Punkten.

\begin{itemize}
	\item \underline{Variante A:} Hier ist ein etwas kleineres Display in Verwendung, genauer gesagt ein 3,5-Zoll Display, mit einem Chipsatz, der anders als bei den anderen Varianten ist, nämlich der Arduino Mega. Für diese Umsetzung muss eine Platine zwischen Display und Board erstellt werden für die Pins, demnach ist es nicht besonders effizient erweiterbar wie beispielsweise bei dem Raspberry PI. Bei dieser Version wird alles in einem IP66 geschützten Gehäuse verbaut und mit Buttons gesteuert. Laut Simon Köldorfer sei diese Methode sinnvoller und einfacher als bei der Nutzung eines Touchscreens, da dieser zu viele Funktionen für eine schlichte Anzeige habe und auch nicht sehr geeignet für bestimmte Wetterkonditionen sei. \\
	Aufgrund der begrenzten Verfügbarkeit von Libraries für die Benutzeroberfläche auf dem Arduino wird das Erscheinungsbild mit der vom Projektteam ausgewählten Library weniger ansprechend sein. Diese Library bietet lediglich die Möglichkeit, mit Texten, Rechtecken, Linien und Kreisen eine Benutzeroberfläche zu kreieren.
	\item \underline{Variante B:} Diese Kombination an Hardware-Komponenten ist der Favorit des Projektteams. Hier wird ein größeres Display mit 7-Zoll und einem Raspberry PI 3 als Chipsatz verwendet. Die Gesamtheit wird wieder in einer IP66-tauglichen Box verstaut. Diese Variante ist zwar teurer als die erste, dafür kann die Umsetzung schöner realisiert werden. Auch kann man sich die \gls{gls_rs485}-Schnittstelle sparen, da man stattdessen einen USB-Port zur Verfügung hat.
	\item \underline{Variante C:} Variante C ist gleich zu Variante A, nur dass ein Touchdisplay implementiert wird, wodurch auch Einsparungen am Gehäuse erfolgen. Diese Einsparung kann jedoch zu Problemen mit bestimmten Witterungen wie zu starker Sonneneinstrahlung, Regen, usw. führen. 
    Es gibt zwar auch andere, widerstandsfähigere Industriedisplays, diese sind aber auch deutlich teurer und bereits ohne das handelt es sich hier um die teuerste Variante, die zusätzlich nicht wasserfest ist.
	\item \underline{Variante D:} Hier handelt es sich um dasselbe Konzept wie in Variante C, nur dass eine Klappe implementiert ist. Diese muss immer geöffnet werden, wenn durch die Werte gescrollt werden will, was nicht sehr praktisch ist, da diese bei zu ofter Verwendung kaputtgehen kann. Auch ist hier die Witterung wieder ein Thema, da das Display bei Regen beispielsweise durch die nicht vorhandene Wasserfestigkeit einen Schaden erhalten kann.
\end{itemize}

\paragraph{Endgültige Entscheidung}
Die Entscheidung gegen die Varianten C und D wurde aufgrund der unnötig komplizierten Handhabung des Touchdisplays, der erhöhten Fehleranfälligkeit und der damit verbundenen hohen Kosten getroffen. \\
Für Variante A wurde angemerkt, dass die einfache Darstellung der Werte möglicherweise nicht ästhetisch ansprechend wäre. Allerdings wurde darauf hingewiesen, dass die Herstellung der Platine von einem derzeitigen Ferialpraktikanten der HTL Rankweil durchgeführt werden könnte. \\
Bezüglich Variante B wurde festgestellt, dass die Umsetzung auch mit einem Raspberry PI Zero möglich wäre, was zusätzliche Kosteneinsparungen bedeutet. Da der Raspberry PI zudem erweiterbar ist und diese Version ein größeres Display verwendet, fiel die Entscheidung auf diese Variante.
