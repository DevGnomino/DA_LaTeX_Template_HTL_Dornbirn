\newpage
\subsection{Selektion} \label{selektion}
Die nach Recherche erstellten Varianten im Kapitel \ref{evaluierung}, mussten innerhalb eines Meetings dem Projektbetreuer und einer weiteren Person als Beisitz vorgestellt werden, damit die Varianten auseinandergenommen werden können und auf den Grund gegangen werden kann, welche für die Umsetzung tatsächlich sinnvoll ist und auch auf wirtschaftlicher Ebene vernünftig wäre. Im Folgenden befindet sich eine kurze Beschreibung der jeweiligen Varianten in Kombination mit den im Meeting angemerkten Punkten.

\begin{itemize}
	\item \underline{Variante A:} Hier wäre ein etwas kleineres Display in Verwendung, genauer gesagt ein 3.5-Zoll-Display, mit einem Chipsatz, der anders als bei den anderen Varianten ist, nämlich der Arduino Mega. Für diese Umsetzung müsste eine Platine zwischen Display und Board erstellt werden für die Pins, demnach ist es nicht besonders effizient erweiterbar wie beispielsweise bei dem Raspberry Pi. Bei dieser Version würde alles in ein IP66 geschütztes Gehäuse verbaut werden und mit Buttons gesteuert. Laut Simon Köldorfer sei diese Methode sinnvoller und einfacher als bei der Nutzung eines Touchscreens, da dieser zu viele Funktionen für eine schlichte Anzeige habe und auch nicht sehr geeignet für Witterungen sei. \\
	Da bei dem Arduino nicht sehr viele Libraries für die Benutzeroberfläche zur Verfügung stehen, würde es mit der vom Projektteam gewählten Library weniger schön aussehen, daher, dass diese eine günstige Variante ist, mit der nur Texte, Rechtecke, Linien und Kreise erstellt werden können.
	
	\item \underline{Variante B:} Diese Kombination an Hardware-Komponenten ist der Favorit des Projektteams. Hier wird ein größeres Display mit 7-Zoll und einem Raspberry Pi 3 als Chipsatz verwendet. Die Gesamtheit würde wieder in einer IP66-tauglichen Box verstaut werden. Diese Variante wäre zwar teurer als die 1., dafür könnte die Umsetzung schöner realisiert werden. Auch könnte man sich die RS323-Schnittstelle sparen, da man stattdessen einen USB-Port zur Verfügung hat.
	\item \underline{Variante C:} Variante C ist gleich zu Variante A, nur dass ein Touchdisplay implementiert werden würde, wodurch auch Einsparungen am Gehäuse erfolgen. Diese Einsparung könnte jedoch zu Problemen mit Witterungen wie zu starker Sonneneinstrahlung, Regen, usw. führen. Natürlich gibt es andere, widerstandsfähigere Industriedisplays, diese sind aber auch deutlich teurer und bereits ohne das handelt es sich hier um die teuerste Variante, die zusätzlich nicht wasserfest ist.
	\item \underline{Variante D:} Hier handelt es sich um dasselbe Konzept wie in Variante C, nur dass eine Klappe implementiert ist. Diese müsste immer geöffnet werden, wenn durch die Werte gescrollt werden will, was nicht sehr praktisch ist, da diese kaputtgehen könnte bei zu öfter Verwendung. Auch ist hier die Witterung wieder ein Thema, da das Display bei Regen beispielsweise durch die nicht vorhandene Wasserfestigkeit einen Schaden erhalten könnte.
\end{itemize}

\paragraph{Endgültige Entscheidung}
Durch die überflüssig komplizierte Handhabung des Displays, die unter Verwendung eines Touchdisplays entstehen würde, die zu große Fehleranfälligkeit und die zu hohen anfallenden Kosten, wurden Variante C und D nicht in Erwägung gezogen. \\
Bei Variante A wurde die einfache Darstellung der Werte bemängelt, da es möglicherweise nicht ästhetisch ansprechend wäre. Bezüglich der herzustellenden Platine wurde dazu jedoch angemerkt, dass diese von dem derzeit in der Firma arbeitenden Ferialpraktikanten aus der HTL Rankweil erstellt werden könnte. \\
Zu Variante B wurde gesagt, dass hier die Umsetzung auch mit einem Raspberry Pi Zero funktionieren würde, was weitere Einsparungen bedeuten würde und da der Raspberry Pi auch noch erweiterbar ist und die Version B ein größeres Display hat, wurde sich für diese Variation entschieden.
